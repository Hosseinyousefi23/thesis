

\فصل{مقدمه}

یادگیری ماشین کاربردی از هوش مصنوعی (AI) است که به کامپیوترها، توانایی یادگیری خودکار از طریق تجربه  را می دهد، بدون اینکه مسنقیما برنامه ریزی شود. حوزه یادگیری ماشینی بر توسعه برنامه های رایانه ای متمرکز است که می توانند به داده ها دسترسی پیدا کنند و از آنها برای یادگیری استفاده کنند. فرایند یادگیری با مشاهده داده ها یا تجربه مستقیم  به منظور کشف الگوهای نهفته در داده ها آغاز می شود؛  سپس از الگوهای  استخراج شده برای اتخاذ تصمیمات بهتری در آینده استفاده می\nf شود. هدف اصلی این است که به کامپیوترها اجازه دهیم بدون دخالت یا کمک انسان و به طور خودکار یاد بگیرند و اقدامات را بر اساس آن تنظیم کنند.

\section{دسته بندی روش های یادگیری ماشین}
الگوریتم های یادگیری ماشین، به طور کلاسیک به دو دسته کلی یادگیری نظارت شده و یادگیری بدون نظارت تقسیم می شوند. بعدها دو دسته جدید یادگیری نیمه نظارت شده و یادگیری تقویتی به این مجموعه اضافه شدند.
\paragraph{یادگیری نظارت شده}

یکی از مهم\nf ترین روش\nf های یادگیری ماشین، یادگیری نظارت\nf شده است. الگوریتم\nf های یادگیری ماشین نظارت \nf شده می\nf توانند  با استفاده از داده\nf های برچسب\nf دار، الگوهایی  را فرابگیرند و سپس آنچه را که آموخته اند، برای پیش بینی برچسب داده های جدید به کار ببندند. الگوریتم\nf های یادگیری نظارت\nf شده، با تجزیه و تحلیل مجموعه داده\nf های برچسب\nf دار، یک تابع تقریبی را برای پیش بینی مقادیر خروجی می\nf آموزد. این الگوریتم، همچنین می\nf تواند خروجی خود را با خروجی صحیح، مقایسه کند تا بتواند، بر این اساس مدل را اصلاح کند.
%Supervised machine learning algorithms can apply what has been learned in the past to new data using labeled examples to predict future events. Starting from the analysis of a known training dataset, the learning algorithm produces an inferred function to make predictions about the output values. The system is able to provide targets for any new input after sufficient training. The learning algorithm can also compare its output with the correct, intended output and find errors in order to modify the model accordingly.

\paragraph{یادگیری بدون نظارت}

در مقابل  یادگیری نظارت\nf شده، الگوریتم\nf های یادگیری ماشین بدون نظارت قرار دارند. این روش\nf ها هنگامی استفاده می\nf شوند که اطلاعات مورد استفاده برای آموزش نه طبقه بندی شده و نه دارای برچسب باشند. در یادگیری بدون نظارت، سیستم\nf ها می\nf توانند تابعی را برای توصیف ساختار پنهان داده\nf های بدون برچسب بیاموزند. این سیستم\nf ها خروجی صحیح را کشف نمی\nf کنند، اما داده\nf ها را کاوش می\nf کنند و می\nf توانند نتایجی را از مجموعه داده\nf ها استخراج کنند.

%In contrast, unsupervised machine learning algorithms are used when the information used to train is neither classified nor labeled. Unsupervised learning studies how systems can infer a function to describe a hidden structure from unlabeled data. The system doesn’t figure out the right output, but it explores the data and can draw inferences from datasets to describe hidden structures from unlabeled data.

\paragraph{یادگیری نیمه نظارت شده}

الگوریتم\nf های یادگیری ماشین نیمه نظارت\nf شده در جایی بین یادگیری نظارت\nf شده و بدون نظارت قرار می\nf گیرند، آن\nf ها از هر دو داده برچسب\nf دار و بدون برچسب برای آموزش استفاده می\nf کنند. داده\nf های مورد استفاده، به طور معمول شامل مقدار کمی از داده\nf های برچسب\nf دار و مقدار زیادی از داده\nf های بدون برچسب است. معمولاً یادگیری نیمه نظارت\nf شده زمانی انتخاب می\nf شود که یادگیری نظارت\nf شده به دلایل مختلف عملی نباشد. این دلایل، می\nf تواند کوچک یا ناکافی بودن داده\nf های برچسب\nf دار، یا پیچیدگی تابع مورد نظر باشد. سیستم\nf هایی که از این روش استفاده می\nf کنند، می\nf توانند به میزان قابل توجهی دقت یادگیری را نسبت به روش نظارت\nf شده بهبود بخشند. 
%Semi-supervised machine learning algorithms fall somewhere in between supervised and unsupervised learning, since they use both labeled and unlabeled data for training – typically a small amount of labeled data and a large amount of unlabeled data. The systems that use this method are able to considerably improve learning accuracy. Usually, semi-supervised learning is chosen when the acquired labeled data requires skilled and relevant resources in order to train it / learn from it. Otherwise, acquiring unlabeled data generally doesn’t require additional resources.	

\paragraph{یادگیری تقویتی}
یادگیری تقویتی، یک روش یادگیری است که از طریف تعامل با محیط و تجربه می \nf آموزد و خطاها یا پاداش ها را کشف می کند. \مهم{آزمون و خطا} و \مهم{تأخیر در پاداش}، مهمترین ویژگی  های یادگیری تقویتی هستند. این روش به عامل اجازه می دهد تا رفتار ایده آل، یعنی رفتاری که مقیاس عملکرد را به حداکثر می\nf رساند را به طور خودکار پیدا کند. سینگال پاداش،  که توسط محیط تولید می\nf شود، این امکان را برای عامل فراهم می\nf کند که بهترین اقدامات را بباموزد. در ادامه این پایان نامه، به بررسی دقیق تر الگوریتم\nf های یادگیری تقویتی خواهیم پرداخت.
%Reinforcement machine learning algorithms is a learning method that interacts with its environment by producing actions and discovers errors or rewards. Trial and error search and delayed reward are the most relevant characteristics of reinforcement learning. This method allows machines and software agents to automatically determine the ideal behavior within a specific context in order to maximize its performance. Simple reward feedback is required for the agent to learn which action is best; this is known as the reinforcement signal.


\section{تاریخچه یادگیری تقویتی}
تاریخچه اولیه یادگیری تقویتی، دو مسیر اصلی دارد که هر دو طولانی و بسیار غنی هستند. یکی از این مسیرها به حوزه \مهم{یادگیری توسط
آزمون و خطا} مربوط می\nf شود که پیدایش آن به حوزه روانشناسی یادگیری حیوانات برمی\nf گردد. این مسیر در اولین مطالعات حوزه هوش مصنوعی به وجود آمد و در اواخر دهه 1980 نیز منجر احیای مجدد هوش مصنوعی شد. مسیر دیگر مربوط به مسئله \مهم{کنترل بهینه} و حل آن با استفاده از توابع ارزش و برنامه\nf ریزی پویا است.

 اگر چه این دو مسیر تا حد زیادی مستقل از یکدیگر حرکت کرده اند،  اما حول یک موضوع سوم و کمتر مشخص یعنی روش\nf های \مهم{اختلاف زمانی} \LTRfootnote{Temporal Difference} با یکدیگر تلاقی می\nf کنند. هر سه رشته در اواخر دهه 1980 گرد هم آمدند تا زمینه\nf های یادگیری تقویتی مدرن را فراهم کنند.


%The early history of reinforcement learning has two main threads, both long and rich, that were pursued
%independently before intertwining in modern reinforcement learning. One thread concerns learning by
%trial and error that started in the psychology of animal learning. This thread runs through some of
%the earliest work in artificial intelligence and led to the revival of reinforcement learning in the early
%1980s. The other thread concerns the problem of optimal control and its solution using value functions
%and dynamic programming. For the most part, this thread did not involve learning. Although the
%two threads have been largely independent, the exceptions revolve around a third, less distinct thread
%concerning temporal-difference methods such as the one used in the tic-tac-toe example in this chapter.
%All three threads came together in the late 1980s to produce the modern field of reinforcement learning

اصطلاح
\textit{کنترل بهینه}
\LTRfootnote{Optimal control}
در اواخر دهه 1950 برای توصیف مسئله طراحی یک کنترل‌گر برای به حداقل رساندن اندازه‌گیری رفتار \textit{سیستم دینامیکی}\LTRfootnote{Dynamical system} در طول زمان مورد استفاده قرارگرفت. یکی از رویکردهای این مسئله در اواسط دهه 1950 توسط ریچارد بلمن\LTRfootnote{Richard Bellman} و دیگران از طریق گسترش نظریه قرن نوزدهم همیلتون\LTRfootnote{Hamilton} و جاکوبی\LTRfootnote{Jacobi} توسعه یافت. این رویکرد از مفاهیم حالت یک سیستم دینامیکی و یک \textit{تابع ارزش}\LTRfootnote{Value function} برای تعریف یک معادله تابعی استفاده می‌کند؛ که اکنون \textit{معادله بلمن} \LTRfootnote {Bellman equation}نامیده می‌شود. مجموعه روش‌های حل مسائل کنترل بهینه به کمک معادله بلمن به عنوان برنامه‌ریزی پویا شناخته می‌شود. همچنین بلمن نسخه گسسته از مسئله کنترل بهینه را که تحت عنوان \textit{فرایندهای تصمیم‌گیری مارکوف} 
\LTRfootnote{Markov decision process(MDP)}
شناخته می‌شود، معرفی کرد. رونالد هوارد (1960) روش
\lr{Policy Iteration}
را برای MDP‌ ها طراحی کرد. همه این‌ها عناصر اساسی در تئوری و الگوریتم‌های \textit{یادگیری تقویتی}\LTRfootnote{Reinforcement learning (RL) } مدرن هستند.

در دهه اخیر، الگوریتم های مدرن یادگیری تقویتی  ژرف مانند DQN با موفقیت های فراوان خود، توانسته اند انقلابی در حوزه هوش مصنوعی و یادگیری ماشین به وجود آورند. از نمونه\nf های آن می\nf توان به
 \lr{OpenAI Five}
 و
  \lr{AlphaZero}
   اشاره کرد.
  
  \paragraph{\lr{OpenAI Five}}
    
%\lr{OpenAI Five}
   نام یک پروژه یادگیری ماشین متعلق به شرکت 
   \lr{OpenAI} 
   است که به عنوان تیمی از ربات ها،  در بازی ویدئویی رقابتی پنج به پنج 
\lr{Dota2}
     در مقابل انسان\nf ها بازی می\nf کند.
\lr{OpenAI} 
 یک شرکت آمریکایی تحقیق و توسعه در زمینه هوش مصنوعی (AI) است که با هدف توسعه هوش مصنوعی ایمن به روشی که به نفع بشریت باشد تاسیس شده است. اولین حضور عمومی
  \lr{OpenAI Five}
  در سال 2017 اتفاق افتاد، جایی که در یک بازی زنده مقابل یک بازیکن حرفه\nf ای معروف به Dendi بازی کرد و او را شکست داد. سال بعد، این سیستم تا جایی که به عنوان یک تیم 5 نفره کامل عمل کند، پیشرفت کرده بود. بعد از پشت سر گذاشتن روند آزمایشی، شروع به بازی در برابر تیم های حرفه ای کرد و توانایی خود را در شکست دادن آن\nf ها به نمایش گذاشت. 
  \lr{Dota2}
  یک بازی ویدیویی چند نفره است که در آن دو تیم، هر یک شامل پنج بازیکن با یکدیگر رقابت می\nf کنند. همکاری تیمی، برنامه ریزی، ارزیابی موقعیت و تصمیم گیری از ملزومات اصلی پیروزی در این بازی است.
   
   
   

%
%OpenAI Five is the name of a machine learning project that performs as a team of video game bots playing against human players in the competitive five-on-five video game Dota 2. The system was developed by OpenAI, an American artificial intelligence (AI) research and development company founded with the mission to develop safe AI in a way that benefits humanity. OpenAI Five's first public appearance occurred in 2017, where it was demonstrated in a live one-on-one game against a professional player of the game known as Dendi, who lost to it. The following year, the system had advanced to the point of performing as a full team of five, and began playing against and showing the capability to defeat professional teams.
  
  

\paragraph{\lr{AlphaZero}}
یک برنامه رایانه ای است که توسط شرکت تحقیقاتی هوش مصنوعی
\lr{DeepMind}
 برای تسلط بر بازی\nf های شطرنج، شوگی و 
\lr{Go}
ساخته شده است. این الگوریتم، از روشی مشابه
\lr{AlphaGo Zero}
  استفاده می\nf کند. در تاریخ 5 دسامبر 2017، تیم
\lr{DeepMind}
  با معرفی
\lr{AlphaZero}
   نسخه اولیه\nf ای را منتشر کرد که در طی 24 ساعت آموزش با شکست دادن برنامه های قهرمان جهان در این سه بازی، یعنی
    \lr{Stockfish}
    برای شطرنج،
\lr{elmo}
برای شوگی و نسخه 3 روزه
\lr{AlphaGo Zero}
برای بازی 
\lr{Go}،
به سطحی فوق بشری در این سه بازی دست یافت. 

برای اجرای الگوریتم یادگیری
\lr{AlphaZero}
از واحدهای پردازش تنسور 
\LTRfootnote{Tensor Processing Unit}
یا 
\lr{TPU}
استفاده شده است که برای اجرای برنامه\nf های
\lr{Google}
بهینه شده\nf اند
\cite{silver2017}.
\lr{AlphaZero}
  فقط از طریق 
  \مهم{
\textit{بازی با خود}
  }
\LTRfootnote{Self-Play}
و بدون استفاده از دانش بشری در این سه بازی، به سطحی باورنکردنی رسید. الگوریتم بازی با خود، یک روش یادگیری تقویتی برای بازی های دو یا چند نفره است که در آن، عامل در مقابل یک کپی از خودش بازی می کند و در روند یادگیری پیشرفت می کند. شرکت
\lr{DeepMind}
  برای تولید  بازی ها از ۵ هزار TPU نسل اول و برای یادگیری شبکه های عصبی از ۶۴ TPU نسل دوم استفاده کرده است که همه به طور موازی آموزش می بینند.
پس از چهار ساعت آموزش،
\lr{DeepMind}
  تخمین زده بود که
\lr{AlphaZero}
    در رتبه بندی
    \lr{Elo}
\footnote{سیستم رتبه بندی
\lr{Elo}
	  روشی برای محاسبه سطح مهارت های نسبی بازیکنان در بازی های با جمع صفر مانند شطرنج است. نام آن به نام خالق آن
\lr{Arpad Elo}،
استاد فیزیک مجارستانی-آمریکایی برمی\nf گردد.}
بالاتر از
\lr{Stockfish 8}
  قرار می گیرد. پس از 9 ساعت آموزش، توانست
   \lr{Stockfish 8}
را در یک مسابقه با 100 بازی و با کنترل زمان، شکست دهد (28 برد، 72 تساوی و بدون باخت) 
\cite{silver2017} \cite{knapton2017entire} \cite{superhuman2017}
درحالی\nf که الگوریتم آموزش دیده بر روی یک ماشین واحد با چهار TPU بازی می کرد.

مقاله
\lr{DeepMind}
  در مورد
\lr{AlphaZero}
    در ژورنال
\lr{Science}
در 7 دسامبر 2018 منتشر شد
\cite{silver2018general}.
در سال 2019،
\lr{DeepMind}
مقاله جدیدی را منتشر کرد و در آن
\lr{MuZero}
را معرفی کرد. 
\lr{MuZero}
الگوریتم جدیدی است که می\nf تواند در بازی های Atari و بازی های تخته\nf ای 
\LTRfootnote{Board-Games}
 بدون اطلاع از قوانین بازی و صرفا با بازی کردن، مهارت کسب کند.


به طور معمول، مسئله یادگیری تقویتی را با استفاده از مدل سیستم دینامیکی به ویژه فرآیند تصمیم گیری مارکوف مدل سازی می شود. در فصل دوم با فرآیند تصمیم گیری مارکوف و ویژگی های آن بیشتر آشنا خواهیم شد.

واژه یادگیری تقویتی، همزمان به چند معنی اشاره می کند، اول مسئله یادگیری تقویتی، که در ادامه با آن بیشتر آشنا می شویم،  دوم، مجموعه راه حل ها و روش هایی که برای حل این مسئله ارائه شده و به خوبی کار می کنند و سوم، حوزه یادگیری تقویتی که به مطالعه مسئله یادگیری تقویتی و راه حل های آن می پردازد. برای اینکه این چندمعنایی ابهامی ایجاد نکند، از این به بعد وقتی از واژه یادگیری تقویتی استفاده می کنیم، منظورمان  مورد دوم یا سوم است. برای اشاره به مورد اول از عبارت «مسئله یادگیری تقویتی» استفاده خواهیم کرد. 

یک موضوع مهم در یادگیری تقویتی آینده نگری و برنامه ریزی است. ممکن است انتخاب یک عمل، پاداش آنی کمتری داشته باشد اما در طولانی مدت، موجب کسب پاداش های بهتر شود. 

یک روش یادگیری ماشین تعریف می‌شود که مربوط به نحوه اقدام یک عامل هوشمند \LTRfootnote{Agent} در محیط براساس هدفی مشخص است. یادگیری تقویتی عبارت است از یادگیری اینکه عامل هوشمند چه کاری باید انجام دهد(نحوه انتخاب اقدامات برحسب موقعیت) تا به حداکثر پاداش برسد. این روش تمامی‌ مسائل یک عامل هدفمند را صریحا در تعامل با یک محیط نامشخص بررسی می‌کند. به عامل هوشمند گفته نمی‌شود که چه کارهایی را انجام دهد ، اما درعوض باید کشف کند که کدام اقدامات، بیشترین پاداش را به همراه دارد. در جالب ترین و چالش برانگیزترین موارد ، اقدامات ممکن دارد نه تنها بر پاداش فوری بلکه در وضعیت بعدی محیط، و از طریق آن، بر کلیه پاداش های بعدی تأثیر بگذارد. این دو ویژگی (جستجوی آزمون و خطا و پاداش تأخیری) دو ویژگی مهم تمییز دهنده یادگیری تقویتی از روش های متداول یادگیری ماشین هستند.
اقدامات عامل هوشمند، می‌تواند بر وضعیت آینده محیط تأثیر بگذارد
یادگیری تقویتی یک رویکرد محاسباتی برای درک و خودکار کردن یادگیری و تصمیم‌گیری هدفمند است.
یادگیری تقویتی، یادگیری از  طریق تعامل است که چگونه می‌توان برای رسیدن به یک هدف رفتار کرد.
عامل یادگیری تقویتی و محیط، در طی مراحل زمانی گسسته یا پیوسته با یکدیگر تعامل دارند.

یادگیری تقویتی، روشی برای فهم و مدل سازی یادگیری هدف محور و تصمیم گیری است.

\section{دینامیک عامل-محیط}
%\قسمت{دینامیک عامل-محیط}
حوزه یادگیری تقویتی  دو بازیگر اصلی دارد؛ عامل و محیط. موجود تصمیم گیرنده و آموزنده را عامل یادگیری، یا به اختصار، عامل می‌نامند. قسمتی که عامل با آن تعامل دارد (هر چیزی خارج از عامل)، محیط نامیده می‌شود. در ادبیات کنترل بهینه، معمولا به جای واژه‌های عامل و محیط، از  کنترل\nf کننده 
\LTRfootnote{Controller}
و سیستم کنترل شده \LTRfootnote{Controlled Systen} استفاده می‌شود.
عامل و محیط به طور مداوم با یکدیگر ارتباط برقرار می‌کنند. عامل انتخاب می‌کند که چه اقدامی‌ انجام دهد، محیط به این اقدامات پاسخ می‌دهد و موقعیت جدیدی را به عامل ارائه می‌دهد.
محیط، همچنین مقادیر عددی ویژه ای به نام پاداش  به عامل برمی‌گرداند، که عامل به دنبال به حداکثر رساندن آن است. 

به طور خاص، عامل و محیط در یک توالی زمانی گسسته تعامل می‌کنند، 
$t = 0,1,2,3,...$،
در هر مرحله $t$، عامل، وضعیت محیط  
$S_t \in \EuScript{S}$
را دریافت می‌کند، و بر اساس آن یک عمل 
$A_t \in \EuScript{A}$
را انتخاب می‌کند. در گام بعدی، عامل به عنوان نتیجه عمل خود، یک پاداش عددی $R_{t+1} \in \EuScript{R}$ دریافت می‌کند و خود را در حالت جدید $S_{t+1}$ می‌یابد.
دینامیک عامل-محیط را می‌توان به شکل یک دنباله از حالت‌ها، عمل‌ها و پاداش‌ها به صورت زیر نمایش داد:
$$S_0, A_0, R_1, S_1, A_1, R_2, S_2, A_2, R_3, ...$$.



\قسمت{تعریف مسئله}
مسئله‌ی \مهم{یادگیری تقویتی} در اصل یک مسئله بهینه سازی است. هدف اصلی مسئله، به حداکثر رساندن پاداشی است که از محیط دریافت می‌شود

تعریف دقیق‌تر این مسئله را در فصل دوم خواهیم دید.

\قسمت{اهمیت موضوع}
یادگیری تقویتی
در بسیاری از رشته ها مانند نظریه بازی، نظریه کنترل، تحقیق در عملیات، نظریه اطلاعات، بهینه سازی مبتنی بر شبیه سازی، سیستم های چند عاملی، هوش انبوه و آمار مورد مطالعه قرار می‌گیرد. در ادبیات تحقیق و کنترل عملیات، یادگیری تقویتی را برنامه ریزی تقریبی پویا  \LTRfootnote{Approximate Dynamic Programming} یا برنامه ریزی عصبی پویا  \LTRfootnote{Neuro-dynamic Programming} می‌نامند. مسائل مورد بررسی در یادگیری تقویتی در نظریه کنترل بهینه \LTRfootnote{Optimal Control Theory} نیز مورد بررسی قرار گرفته است، که بیشتر مربوط به وجود و توصیف راه حل های بهینه و الگوریتم های محاسبه دقیق آنهاست، و کمتر مربوط به یادگیری یا تقریب، به ویژه در غیاب یک مدل ریاضی از محیط. در اقتصاد و نظریه بازی، ممکن است از یادگیری تقویتی برای توضیح چگونگی ایجاد تعادل، استفاده شود.


\subsection{تفاوت بین یادگیری تحت نظارت و یادگیری تقویتی}
یادگیری تحت نظارت ، یادگیری از مجموعه آموزش مثالهای برچسب دار است که توسط یک ناظر بیرونی آگاه ارائه می شود. هدف این نوع یادگیری این است که سیستم پاسخهای خود را برون یابی یا تعمیم دهد تا در موقعیت هایی که در مجموعه آموزش وجود ندارد ، به درستی عمل کند. در مسائل تعاملی ، به دست آوردن نمونه هایی از رفتار مطلوب که هم صحیح باشد و هم نمایانگر همه موقعیت هایی که عامل باید عمل کند، غیر عملی است. در قلمرو ثبت نشده (جایی که انتظار می رود یادگیری مفید باشد) یک عامل باید بتواند از تجربیات خود بیاموزد

\subsection{تفاوت بین یادگیری بدون نظارت و یادگیری تقویتی}
یادگیری بدون نظارت معمولاً یافتن ساختاری است که در مجموعه داده های بدون برچسب پنهان شده است. یادگیری تقویتی به جای تلاش برای یافتن ساختار پنهان در تلاش است تا یک سیگنال پاداش را به حداکثر برساند

\subsection{اکتشاف و بهره برداری}
یکی از چالش هایی که در یادگیری تقویتی برخلاف سایر روش‌های یادگیری وجود دارد ، رقابت بین اکتشاف و بهره برداری است. برای به دست آوردن پاداش زیاد ، یک عامل یادگیری تقویتی باید کارهایی را ترجیح دهد که در گذشته انجام داده و در تولید پاداش موثرتر بوده است. اما برای کشف چنین اعمالی، باید اقداماتی را امتحان کند که قبلاً انتخاب نکرده است. این عامل برای به دست آوردن پاداش مجبور است از آنچه قبلاً تجربه کرده است بهره برداری کند، اما همچنین برای انتخاب اقدامات بهتر در آینده باید به کاوش بپردازد. 
اکتشاف و بهره برداری هیچکدام به تنهایی در رسیدن به هدف، کارا نیست.
\subsection{چنگک چند اهرمه}

مسئله چنگک چند اهرمه یکی از ساده ترین مسائل یادگیری تقویتی است که مفهوم اکتشاف و بهره برداری را به خوبی توضیح می دهد. مسئله زیر را در نظر بگیرید. شما به طور مکرر با انتخاب در میان k گزینه یا اقدام مختلف روبرو هستید که از این به بعد به آن ها عمل۱ تا عمل k می گوییم. بعد از هر انتخاب، پاداش عددی دریافت می کنید. پاداش هر عمل از یک توزیع احتمال مجهول ولی ثابت پیروی می کند. ممکن است توزیع احتمال یک عمل با دیگری متفاوت باشد ولی توزیع احتمال ها در طول زمان تغییر نمی کنند. هدف شما کسب بیشترین پاداش، در یک دوره زمانی طولانی است (مثلا بعد از ۱۰۰۰ بار بازی کردن). 
 این مسئله یک استراتژی بدیهی دارد. هر کدام از توزیع ها، یک مقدار میانگین یا امیدریاضی دارند که مقدار پاداشی است که به طور متوسط به ازای هر بار انتخاب شدن، اهدا می کند. فرض کنید امیدریاضی پاداش هر عمل را \مهم{ارزش} آن عمل بنامیم. در این صورت استراتژی بدیهی، انتخاب مکرر عملی است که بیشتری ارزش را دارد. به این عمل، \مهم{عمل بهینه} 
\LTRfootnote{Greedy}
 گفته می شود. به عبارت دیگر در این مسئله، یک عمل بهینه وجود دارد و آن عملی است که نسبت به بقیه اعمال، ارزش بیشتری دارد. با انتخاب مکرر این عمل، به طور میانگین بیشترین پاداش را کسب خواهیم کرد. اما نکته مهم در این مسئله این است که ما ارزش هیج یک از عمل ها را نمی دانیم، بنابراین نمی دانیم که عمل بهینه کدام است. اما می توانیم با انتخاب اعمال مختلف و مشاهده پاداش ها، تخمینی از ارزش هر عمل کسب کنیم. به این کار، \مهم{اکتشاف} گفته می\nf شود. فرض کنید تخمین ما از ارزش یک عمل، میانگین پاداش هایی است که تابه حال به ما داده است. دقت این تخمین، وابسته به دفعاتی است که آن عمل را انتخاب کرده ایم. هر چقدر بیشتر عملی را انتخاب و پاداش آن را مشاهده کنیم، تخمین ما از ارزش آن عمل، به ارزش واقعی آن نزدیک تر خواهد بود. به بیان دیگر، هر چه بیشتر اکتشاف کنیم، تخمین های دقیق تری خواهیم داشت.
%
%این شکل اصلی مشکل چنگک k اهرمه است ، بنابراین به قیاس با دستگاه اسلات نامیده می شود ، یا
%"راهزن یک مسلح" ، با این تفاوت که به جای یک اهرم ، k اهرم دارد. هر انتخاب اکشن مانند یک بازی است
%از اهرم های دستگاه اسلات ، و جوایز بازدهی برای ضربه زدن به برنده تمام پولها است. از طریق تکرار شده است
%گزینه های انتخابی شما باید با تمرکز اقدامات خود بر روی بهترین اهرم ها ، برنده های خود را به حداکثر برسانید.
%تشبیه دیگر این است که پزشک بین یک روش درمانی تجربی برای یک سری موارد جدی انتخاب می کند
%بیماران بیمار هر عملی انتخاب یک درمان است و هر پاداش بقا یا رفاه است
%از بیمار امروزه از اصطلاح "مسئله راهزن" برای تعمیم مسئله استفاده می شود

اگر تخمین ارزش ها را کنیم، در هر مرحله حداقل یک اقدام وجود دارد
که ارزش تخمینی آن بیشینه است. ما اینها را اقدامات \مهم{حریصانه} می نامیم. وقتی یکی از این اقدامات را انتخاب می کنیم،
درواقع از دانش فعلی خود در مورد ارزش اقدامات، \مهم{بهره برداری} می کنیم. در عوض اگر،
یکی از اقدامات کم ارزش تر را انتخاب کنیم، سپس در حال \مهم{اکتشاف} هستیم، زیرا این کار ما را قادر می سازد تا برآوردهای بهتری از ارزش ها به دست آوریم. هدف بهره برداری، به حداکثر رساندن پاداش چشمداشتی در یک مرحله است، اما اکتشاف ممکن است در دراز مدت پاداش کل بیشتری را به بار آورد.
به عنوان مثال، فرض کنید ارزش یک عمل حریصانه با قطعیت شناخته شده باشد ، در حالی که چندین عمل دیگر نیز وجود دارند که
تقریبا به همان خوبی  هستند، اما ارزش تخمینی آن ها با عدم اطمینان قابل توجهی دارد. عدم اطمینان به حدی است که حداقل
یکی از این اقدامات دیگر احتمالاً بهتر از عمل حریصانه است، اما شما نمی دانید کدام یک است.
اگر زمان زیادی پیش رو داریم، بهتر است
اقدامات دیگر را اکتشاف کنیم تا بفهمیم کدام یک از آنها بهتر از عمل حریصانه است. با این کار، احتمالا در کوتاه مدت، یعنی در حین اکتشاف، پاداش کمتری کسب می کنیم، اما در بلند مدت، پاداش بیشتری دریافت خواهیم کرد، زیرا پس از کشف اقدامات بهتر، می توانیم چندین بار از آنها بهره برداری کنیم. 
هیچ یک از اکتشاف و بهره برداری، به تنهایی نمی توانند ما را به بیشترین پاداش چشمداشتی در بلند مدت برسانند، در یک استراتژی مناسب، هر دو اینها باید لحاظ شوند. به دلیل آنکه هم اکتشاف و هم بهره برداری، در یک گام امکان پذیر نیست، ناچاریم در هر مرحله تصمیم بگیریم که کدام یک را انجام دهیم. این مشکل، به \مهم{تعارض} بین اکتشاف و بهره برداری معروف است.
%در هر مورد خاص ، بهتر است اکتشاف یا بهره برداری به روش پیچیده ای به دقت بستگی دارد
%مقادیر تخمین ها ، عدم قطعیت ها و تعداد مراحل باقی مانده. بسیاری از پیچیده وجود دارد
%روشهای متعادل سازی اکتشاف و بهره برداری برای فرمولهای خاص ریاضی راهزن کارمند و مشکلات مربوطه. با این حال ، بیشتر این روشها فرضیات محکمی راجع به
%ثابت بودن و دانش قبلی که نقض شده یا تأیید در برنامه ها و غیرممکن است
%مسئله یادگیری تقویت کامل که در فصل های بعدی در نظر می گیریم. تضمین های
%بهینه بودن یا از دست دادن محدود برای این روشها هنگام فرضیات نظریه آنها چندان راحت نیست
%اعمال نمی شود


\subsection{ربات زباله جمع کن}
یک ربات موبایل وظیفه جمع آوری قوطی های نوشابه خالی در آن را دارد
یک محیط اداری دارای سنسورهایی برای تشخیص قوطی ها و بازو و گیرنده ای است که می تواند آنها را انتخاب کند
آنها را در سطل آشغال قرار دهید. با باتری قابل شارژ کار می کند. سیستم کنترل ربات
دارای م components لفه هایی برای تفسیر اطلاعات حسی ، جهت یابی و کنترل بازو و
گیرنده تصمیمات سطح بالا در مورد چگونگی جستجوی قوطی ها توسط یک عامل یادگیری تقویت کننده اتخاذ می شود
بر اساس میزان شارژ فعلی باتری. این نماینده باید تصمیم بگیرد که آیا ربات باید (1)
برای مدت زمان مشخصی به طور فعال به جستجو در یک قوطی بپردازید ، (2) ثابت بمانید و منتظر بمانید تا کسی چه کار کند
آن را قوطی بیاورید ، یا (3) برای شارژ مجدد باتری خود به پایه خانه خود برگردید. این تصمیم باید گرفته شود
به صورت دوره ای یا هر زمان که حوادث خاصی رخ می دهد ، مانند یافتن قوطی خالی. بنابراین نماینده
دارای سه عملکرد است و حالت در درجه اول با توجه به وضعیت باتری تعیین می شود. جوایز ممکن است
بیشتر اوقات صفر باشید ، اما وقتی ربات قوطی خالی یا بزرگ و بزرگ را محکم کرد ، مثبت می شوید
منفی است اگر باتری تمام کار کند. در این مثال ، عامل یادگیری تقویت کننده نیست
کل ربات ایالات که نظارت می کند شرایط موجود در خود ربات را توصیف می کند ، نه شرایط آن را
محیط خارجی ربات بنابراین محیط عامل شامل بقیه ربات است که
ممکن است شامل سایر سیستم های تصمیم گیری پیچیده و همچنین محیط خارجی ربات باشد.



\subsection{کارهای اپیزودیک و ادامه دار}
گاهی اوقات ، تعامل عامل و محیط به طور طبیعی به قسمت هایی تقسیم می شود که آنها را اپیزود می نامند. هر اپیزود، در یک حالت خاص به نام حالت ترمینال به پایان می رسد و به دنبال آن دوباره از یک حالت شروع استاندارد یا یک نمونه از یک توزیع استاندارد از حالت های شروع، آغاز می شود. هر اپیزود به طور مستقل از اپیزود قبلی آغاز می شود.
از طرف دیگر، در بسیاری از موارد تعامل عامل و محیط به طور طبیعی به قسمت های قابل شناسایی تقسیم نمی شود، بلکه به طور مداوم و بدون محدودیت ادامه دارد. ما این کارها را ادامه دار می‌نامیم.


\section{عناصر اصلی یادگیری تقویتی}
چهار عنصر کلیدی و اصلی مسائل یادگیری تقویتی عبارت است از:
\textit{خط‌مشی} \LTRfootnote{Policy}، 
\textit{سیگنال پاداش} \LTRfootnote{Reward Signal}،
\textit{تابع ارزش} \LTRfootnote{Value Function} 
و
\textit{محیط} \LTRfootnote{Environment}
\subsection{خط‌مشی}
خط‌مشی\LTRfootnote{Policy} نحوه رفتار عامل یادگیری را در یک زمان خاص، مشخص می‌کند و هسته اصلی رفتار یک عامل یادگیری تقویتی است. به بیان دیگر، خط‌مشی، نگاشتی از حالت‌های مدل شده از محیط به عملی است که باید در آن حالت انجام شود. خط‌مشی به تنهایی برای تعیین رفتار عامل کافی است. ممکن است
خط‌مشی عامل یادگیری، یک تابع ساده یا جدول جستجو باشد، یا ممکن است شامل محاسبات پیچیده‌ای مانند فرآیند جستجو باشد؛ همچنین
خط‌مشی‌ها ممکن است احتمالاتی باشند.


\subsection{سیگنال پاداش}

همه ی آنچه به عنوان هدف مدنظرداریم می‌تواند به صورت بیشینه سازی  یک سیگنال عددی بیان شود.

در مسئله یادگیری تقویتی، تصمیمات عامل یادگیری، توسط سیگنال پاداش جهت‌دهی می‌شود. در هر گام، محیط، یک عدد حقیقی به عنوان پاداش برای عامل یادگیری تقویتی ارسال می‌کند. تنها هدف عامل، به حداکثر رساندن کل پاداش دریافتی از محیط در طولانی مدت است. سیگنال پاداش اتفاقات خوب و بد را برای عامل مشخص می‌کند و مبنای اصلی تغییر خط‌مشی است
سیگنال پاداش می‌تواند تابعی تصادفی از وضعیت محیط و عمل انجام شده باشد.

%In reinforcement learning, the purpose or goal of the agent is formalized in terms of a special signal,
%called the reward, passing from the environment to the agent. At each time step, the reward is a simple
%number, Rt ∈ R. Informally, the agent’s goal is to maximize the total amount of reward it receives.
%This means maximizing not immediate reward, but cumulative reward in the long run. We can clearly
%state this informal idea as the reward hypothesis

\subsection{عایدی و تابع ارزش}
همانطور که سیگنال پاداش نشان می‌دهد که انجام چه عملی در هر گام خوب است، تابع ارزش مشخص می\nf کند که کدام عمل در طولانی مدت بهتر است. در واقع تابع ارزش نشانگر مطلوبیت طولانی مدت حالت‌ها با در نظر گرفتن حالت‌هایی است که در پی خواهند داشت.

پاداش‌ها به یک معنا اولیه هستند، در حالی که ارزش‌ها، به عنوان پیش‌بینی پاداش‌ها، ثانویه هستند. بدون پاداش هیچ ارزشی وجود ندارد و تنها هدفِ تخمین ارزش‌ها، دستیابی به پاداش بیشتر است. با این وجود، این تابع ارزش است که هنگام تصمیم‌گیری و ارزیابی به آن توجه می‌کنیم.
تعیین ارزش،‌ بسیار دشوارتر از تعیین پاداش است.
ارزش‌ها باید از توالی مشاهداتی که یک عامل در طول عمر خود انجام می‌دهد، برآورد شوند.
مهمترین مولفه بیشتر الگوریتم‌های یادگیریِ تقویتی که در  فصل های آینده معرفی خواهیم کرد، روشی برای تخمین کارآمد تابع ارزش است.

\subsection{مدل محیط}

مدلِ محیط رفتار محیط را تقلید می‌کند، یا به طور کلی‌تر، اجازه می‌دهد تا در مورد نحوه رفتار محیط، پیشبینی کارآمدی داشته‌باشیم. از مدل‌ها برای برنامه‌ریزی و انتخاب در روند تصمیم‌گیری  با در نظر گرفتن شرایط احتمالی آینده بدون تجربه واقعی آن‌ها استفاده می‌شود.
روش‌هایی که برای حل مشکلات یادگیری تقویتی از مدل‌ محیط استفاده می‌کنند، روش‌های
 \textit{مبتنی بر مدل}
\LTRfootnote{Model Based}
نامیده می‌شوند. در مقابل این روش‌ها، روش‌های 
\textit{بدون مدل}
\LTRfootnote{Model Free}
هستند که از هیچ گونه شبیه سازی  یا ابزاری برای پیشبینی رفتار محیط استفاده نمی‌کنند و معمولا روند یادگیری آن‌ها بر اساس تجربه واقعی در محیط و آزمون و خطا است.


\قسمت{فضای حالت\nf ها و عمل\nf ها}

\قسمت{ساختار پایان‌نامه}

این پایان‌نامه شامل --- فصل است. در فصل ...
