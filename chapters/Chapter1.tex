

\فصل{مقدمه}

یک روش یادگیری ماشین تعریف می‌شود که مربوط به نحوه اقدام یک عامل هوشمند \LTRfootnote{Agent} در محیط براساس هدفی مشخص است. یادگیری تقویتی عبارت است از یادگیری اینکه عامل هوشمند چه کاری باید انجام دهد(نحوه انتخاب اقدامات برحسب موقعیت) تا به حداکثر پاداش برسد. این روش تمامی‌ مسائل یک عامل هدفمند را صریحا در تعامل با یک محیط نامشخص بررسی می‌کند. به عامل هوشمند گفته نمی‌شود که چه کارهایی را انجام دهد ، اما درعوض باید کشف کند که کدام اقدامات، بیشترین پاداش را به همراه دارد. در جالب ترین و چالش برانگیزترین موارد ، اقدامات ممکن دارد نه تنها بر پاداش فوری بلکه در وضعیت بعدی محیط، و از طریق آن، بر کلیه پاداش های بعدی تأثیر بگذارد. این دو ویژگی (جستجوی آزمون و خطا و پاداش تأخیری) دو ویژگی مهم تمییز دهنده یادگیری تقویتی از روش های متداول یادگیری ماشین هستند.
اقدامات عامل هوشمند، می‌تواند بر وضعیت آینده محیط تأثیر بگذارد
یادگیری تقویتی یک رویکرد محاسباتی برای درک و خودکار کردن یادگیری و تصمیم‌گیری هدفمند است.
یادگیری تقویتی، یادگیری از  طریق تعامل است که چگونه می‌توان برای رسیدن به یک هدف رفتار کرد.
عامل یادگیری تقویتی و محیط، در طی مراحل زمانی گسسته یا پیوسته با یکدیگر تعامل دارند.
\قسمت{تعریف مسئله}
مسئله‌ی \مهم{یادگیری تقویتی} در اصل یک مسئله بهینه سازی است. هدف اصلی مسئله، به حداکثر رساندن پاداشی است که از محیط دریافت می‌شود

تعریف دقیق‌تر این مسئله را در فصل دوم خواهیم دید.

\قسمت{اهمیت موضوع}
یادگیری تقویتی
در بسیاری از رشته ها مانند نظریه بازی، نظریه کنترل، تحقیق در عملیات، نظریه اطلاعات، بهینه سازی مبتنی بر شبیه سازی، سیستم های چند عاملی، هوش انبوه و آمار مورد مطالعه قرار می‌گیرد. در ادبیات تحقیق و کنترل عملیات، یادگیری تقویتی را برنامه ریزی تقریبی پویا  \LTRfootnote{Approximate Dynamic Programming} یا برنامه ریزی عصبی پویا  \LTRfootnote{Neuro-dynamic Programming} می‌نامند. مسائل مورد بررسی در یادگیری تقویتی در نظریه کنترل بهینه \LTRfootnote{Optimal Control Theory} نیز مورد بررسی قرار گرفته است، که بیشتر مربوط به وجود و توصیف راه حل های بهینه و الگوریتم های محاسبه دقیق آنهاست، و کمتر مربوط به یادگیری یا تقریب، به ویژه در غیاب یک مدل ریاضی از محیط. در اقتصاد و نظریه بازی، ممکن است از یادگیری تقویتی برای توضیح چگونگی ایجاد تعادل، استفاده شود.

\قسمت{فضای حالت\nf ها و عمل\nf ها}
\قسمت{خط مشی}
\قسمت{سیگنال پاداش}
%[Suttonbook]
In reinforcement learning, the purpose or goal of the agent is formalized in terms of a special signal,
called the reward, passing from the environment to the agent. At each time step, the reward is a simple
number, Rt ∈ R. Informally, the agent’s goal is to maximize the total amount of reward it receives.
This means maximizing not immediate reward, but cumulative reward in the long run. We can clearly
state this informal idea as the reward hypothesis

همه ی آنچه به عنوان هدف مدنظرداریم می‌تواند به صورت بیشینه سازی مقدار میانگین یک سیگنال عددی بیان شود.



\قسمت{تابع ارزش}
\قسمت{محیط}
\قسمت{مسئله اکتشاف و بهره برداری}
یکی از چالش هایی که در یادگیری تقویتی برخلاف سایر روش‌های یادگیری وجود دارد ، رقابت بین اکتشاف و بهره برداری است. برای به دست آوردن پاداش زیاد ، یک عامل یادگیری تقویتی باید کارهایی را ترجیح دهد که در گذشته انجام داده و در تولید پاداش موثرتر بوده است. اما برای کشف چنین اعمالی، باید اقداماتی را امتحان کند که قبلاً انتخاب نکرده است. این عامل برای به دست آوردن پاداش مجبور است از آنچه قبلاً تجربه کرده است بهره برداری کند، اما همچنین برای انتخاب اقدامات بهتر در آینده باید به کاوش بپردازد.
اکتشاف و بهره برداری هیچکدام به تنهایی در رسیدن به هدف، کارا نیست.




\قسمت{اهداف تحقیق}


\قسمت{ساختار پایان‌نامه}

این پایان‌نامه شامل --- فصل است. در فصل ...
