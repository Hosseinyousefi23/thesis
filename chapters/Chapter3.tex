
\فصل{معرفی چند الگوریتم جدید}
در فصل دوم دیدیم که مسئله یادگیری تقویتی، در اصل یک مسئله بهینه سازی است که هدف آن یافتن یک خط\nf مشی بهینه (یا دست\nf کم تقریبی از آن) است که در یک فرآیند تصمیم\nf گیری مارکوف تعریف می\nf شود. همچنین  با معادله بلمن و روش\nf های برنامه‌ریزی پویا آشنا شدیم. در این فصل، دو رویکرد جدید به مسئله خواهیم داشت که امکان استفاده از روش\nf های بهینه سازی شناخته شده مانند 
\textit{\مهم{صعود گرادیان}}\LTRfootnote{Gradient Ascent}
و
\textit{\مهم{نزول گرادیان}}\LTRfootnote{Gradient Descent}
را برای حل مسئله یادگیری تقویتی فراهم می\nf کند. سپس برخی از مهم\nf ترین الگوریتم‌های یادگیری تقویتی جدید، مانند
\lr{DQN}،
\lr{DDPG},
\lr{TRPO} و 
\lr{Actor-Critic}
  را معرفی خواهیم کرد. 
\قسمت{دو رویکرد مختلف به مسئله}
 در فصل گذشته  دیدیم که با تعریف یک ترتیب جزئی روی خط\nf مشی\nf های ممکن برای یک
\lr{MDP}،
می\nf توان تعبیری برای خط\nf مشی بهینه ارائه کرد. در این فصل، تعریف دیگری از خط مشی بهینه ارائه خواهد شد.  نشان دادیم که می\nf توان با محاسبه تابع ارزش یک خط\nf مشی، آن را بهبود بخشید. الگوریتم تکرار خط\nf مشی، با شروع از یک خط مشی تصادفی و بهبود دادن آن به صورت گام\nf به\nf گام، به یک خط\nf مشی بهینه همگرا می\nf شود.  به طور کلی به روش\nf هایی که با استفاده از بهبود گام\nf به\nf گام خط\nf مشی، مستقیما خط\nf مشی بهینه  را تقریب می\nf زنند، روش\nf های 
\textit{\مهم{مبتنی بر خط\nf مشی}}
\LTRfootnote{Policy Based}
گفته می\nf شود. بنابراین روش تکرار خط\nf مشی، که در فصل قبل معرفی شد، یک روش مبتنی بر خط\nf مشی است.
 همچنین دیدیم که همه خط\nf مشی\nf های بهینه، تابع ارزش مشترکی دارند که تابع ارزش بهینه نامیده  می\nf شود و در معادله\nf ای موسوم به معادله بهینگی بلمن صدق می\nf کند.
\شروع{تعریف}[تابع مقیاس عملکرد] فرض کنید 
$\EuScript{M} = \seq{\EuScript{S},\EuScript{A},\EuScript{R},\EuScript{P}, \lambda}$ 
یک فرآیند تصمیم\nf گیری مارکوف و $\Pi$ مجموعه تمام خط\nf مشی\nf های ممکن برای $\EuScript{M}$ باشد. تابع 
$J: \Pi \to \mathbb{R}$
به شکل زیر تعریف می\nf شود
$$J(\pi) \triangleq \mathbb{E}_\pi(G_0)$$
\پایان{تعریف}
منظور از 
$G_0$،
عایدی در زمان 
$t=0$
است. از تعریف 
\ref{def:returninf}
داریم
$$G_0 = \sum_{k=0}^{\infty} \gamma^{k} R_{k+1} = R_1 + \gamma R_2 + \gamma^2 R_3 + ...$$
به تابع $J$، تابع مقیاس عملکرد خط\nf مشی گفته می\nf شود.
خط\nf مشی بهینه، خط\nf مشی\nf ای است که مقیاس عملکرد $J$ را بیشینه می\nf کند
$$arg \max_{\pi} J(\pi) = \Pi_*.$$
برخی از الگوریتم\nf های یادگیری تقویتی، روشی برای پیداکردن نقطه بیشینه تابع $J$ (یا تقریبی از آن) ارائه می\nf دهند. روش\nf هایی که چنین رویکردی دارند، روش\nf های 
\textit{\مهم{مبتنی بر خط\nf مشی}}\LTRfootnote{Policy Based}
 نامیده می\nf شوند.
 
 رویکرد دیگری که برای حل مسئله یادگیری تقویتی وجود دارد، محاسبه تابع ارزش حالت، یا تابع ارزش عمل بهینه است. منظور از تابع ارزش بهینه، تابع ارزشی است که در معادله بهینگی بلمن 
 \ref{eq:bellman-opti}
 صدق کند. برخی از روش\nf های یادگیری تقویتی، روشی برای محاسبه تابع ارزش بهینه با استفاده از معادله بهینگی بلمن ارائه می\nf دهند. به روش\nf هایی که چنین رویکردی دارند، روش\nf های 
 \textit{
 	\مهم{مبتنی بر ارزش
 		}}\LTRfootnote{Value Based}
 گفته می\nf شود.

\section{روش‌های مبتنی بر خط‌مشی}
در این قسمت روش‌هایی را در نظر می‌گیریم که برای دستیابی به خط‌مشی بهینه، به جای استفاده از تابع ارزشِ عمل یا ارزشِ حالت، یک خط‌مشی پارامتری‌شده\LTRfootnote{Parameterized Policy} 
با پارامترهای $\theta$ را می‌آموزد.  این کار  معمولا توسط یک شبکه عصبی با پارامترهای 
$\theta \in \mathbb{R}^{d'}$
 به عنوان تقریب\nf گر خط\nf مشی 
$\pi$
شناخته شده و با نماد $\pi_\theta$ نمایش داده می\nf شود. احتمال انتخاب عمل $a$ در حالت $s$ توسط خط\nf مشی $\pi$ با پارامترهای $\theta$ را به صورت
$\pi_\theta(a|s)$
یا
$\pi(a|s;\theta)$
نمایش خواهیم داد.
ممکن است برای یادگیری پارامترهای خط‌مشی از یک تابع ارزش نیز استفاده شود، اما تابع ارزش برای انتخاب عمل مورد نیاز نیست. 
%$\pi_\theta(a|s)$
پارامترهای $\theta$ با استفاده از \مهم{صعود گرادیان}
\LTRfootnote{Gradian ascent}
روی مقیاس عملکرد\LTRfootnote{Performance Measure} 
$J(\pi_\theta)$
(یا به اختصار $J(\theta)$)
به طور مستقیم و یا با بیشینه‌سازی تخمین‌های محلی از  $J(\theta)$ به\nf روزرسانی می‌شوند.
این روش\nf ها تقریبا همیشه به صورت بر-خط\nf مشی  عمل می‌کنند. 
 هدف این روش‌ها بیشینه کردن مقیاس عملکرد است. 
 
 
 \subsection{روش گرادیان خط\nf مشی}
روش گرادیان خط\nf مشی، ساده\nf ترین و پایه\nf ای\nf ترین روش، در میان روش\nf های مبتنی بر خط\nf مشی است. در این روش، دنباله\nf 
ای از پارامترهای خط\nf مشی، 
$\theta_0, \theta_1, \theta_2, ...$
تولید می\nf شود که $\theta_0$ به صورت تصادفی مقداردهی می\nf شود و از جمله 
$\theta_1$
به بعد، هر جمله، از روی جمله قبلی توسط قانون به\nf روزرسانی زیر به\nf دست می\nf آید
\begin{align}
\theta_{t+1} \doteq \theta_t + \alpha \widehat{\nabla J(\theta_t)}.
\label{eq:updatepg}
\end{align}
که 
$\widehat{\nabla J(\theta_t)}$
تخمینی نااریب از گرادیان  $J$ نسبت به 
$\theta_t$
 می\nf باشد که از نمونه تجربه شده به\nf دست می\nf آید و $\alpha$ یک عدد حقیقی مثبت است که 
 \textit{ضریب یادگیری}\LTRfootnote{Learning Coefficient}
 نامیده می\nf شود. می\nf توان نشان داد که دنباله خط\nf مشی\nf های
 $\pi_{\theta_0}, \pi_{\theta_1}, \pi_{\theta_2}, ...$
 به یک خط مشی بهینه $\pi_*$ همگرا می\nf شود. سرعت این همگرایی، به واریانس تخمینگر گرادیان وابسته است، هر چه واریانس بزرگتر باشد، همگرایی دیرتر اتفاق می\nf افتد. به روش‌هایی که چنین الگویی را برای محاسبه خط‌مشی بهینه دنبال می‌کنند، روش‌های \textit{گرادیان خط‌مشی }\LTRfootnote{Policy Gradient} گفته می\nf شود. 

\شروع{قضیه}[گرادیان خط‌مشی] این قضیه بیان می\nf کند که
\begin{align}
	\nabla J(\theta) \propto \mathbb{E}_\pi \left[ G_t \frac{\nabla_\theta \pi (A_t|S_t; \theta)}{\pi(A_t|S_t); \theta} \right]
	\label{eq:pg}
\end{align}
%\sum_{s} \mu(s) \sum_{a} q_\pi(s,a) \nabla_\theta \pi_\theta(a|s)
%که $\mu$ یک توزیع احتمال روی 
%$\EuScript{S}$
% است که متناسب با تعداد دفعاتی است که حالت $s$ با دنبال کردن خط‌مشی 
%$\pi_\theta$
%تکرار می‌شود.
\برچسب{th:pg}
\پایان{قضیه}
\شروع{اثبات}
رجوع شود به 
\cite{suttonbook}
\پایان{اثبات}

\subsection{الگوریتم REINFORCE}

یک نمونه از روش‌های بهینه‌سازی خط‌مشی، خانواده
\lr{REINFORCE}
از الگوریتم‌های یادگیری تقویتی است
\cite{williams1992simple}.
الگوریتم استاندارد \lr{REINFORCE} پارامترهای $\theta$ را در جهت 
$G_t \nabla_\theta  log  \pi (a_t|s_t;\theta)$ به‌روزرسانی می‌کند که یک تخمین نااریب از
$\nabla_\theta J(\theta)$
است.
%می‌توان نشان داد 
%$$\nabla J(\pi_\theta) = \mathbb{E}_\pi \left[ R_t \frac{\nabla_\theta \pi_\theta (a|S_t)}{\pi_\theta (a|S_t)} \right]$$
با توجه به قضیه 
\ref{th:pg}
در گام $t$،
$\left[ G_t \frac{\nabla_\theta \pi_\theta (a|S_t)}{\pi_\theta (a|S_t)} \right]$
یک تخمین‌گر نااریب از $\nabla J(\pi_\theta)$ خواهد بود \cite{suttonbook}. پس می‌توان در هر گام، پارامترهای خط\nf مشی
 را به شکل زیر به‌روزرسانی کرد
$$\theta_{t+1} \doteq \theta_t + \alpha G_t \frac{\nabla_\theta \pi_\theta (a|S_t)}{\pi_\theta (a|S_t)} = \theta_t + \alpha G_t \nabla_\theta log \pi_\theta (a|S_t)$$

\شروع{الگوریتم}{الگوریتم دوره\nf ای 
\lr{REINFORCE} }
\ورودی{یک تابع پارامتری مشتق پذیر از خط‌مشی 
	$\pi(a|s;\theta)$}

\دستور{پارامترهای خط‌مشی 
	$\theta \in \mathbb{R}^{d'}$
	را مقداردهی اولیه کن
}
\به‌ازای{تکرار کن}
%\اگر{$|E| > 0$}
%	\دستور{یک کاری انجام بده}
%\پایان‌اگر
\دستور{یک دوره 
	$S_0,A_0,R_1,..., S_{T-1}, A_{T-1}, R_T$
	 را با دنبال کردن خط\nf مشی $\pi_\theta$ بساز}

\به‌ازای{برای هر گام
$t=0,...,T-1$}
\دستور{
$G_t$
$\longrightarrow$
	 عایدی گام $t$ ام
	 


}
\دستور{
$\theta + \alpha \gamma^t G_t \nabla_\theta ln \pi (A_t|S_t;\theta) \longrightarrow \theta$
}
\پایان‌به‌ازای
\پایان{الگوریتم}
%چند نمونه از روش‌های بهینه‌سازی خط‌مشی به شرح زیر است. \\
%روش‌های بازیگر منتقد
%\LTRfootnote{Actor-Critic}
%که الگوریتم 
%گرادیان افزایشی
%را مستقیما برای بیشینه سازی 
%$J(\pi\theta)$
%به کار می‌برند.
%روش  
%\lr{Proximal Policy Optimization}
%که!!!!!!!!!!!

\subsubsection{روش‌های بازیگر-منتقد}
در بخش قبل، با روش گرادیان خط\nf مشی و یک روش عملی آن یعنی الگوریتم
\lr{REINFORCE}
آشنا شدیم. دسته‌ای از روش‌های گرادیان خط‌مشی وجود دارند که علاوه بر بهبود گام به گام خط \nf مشی، تخمینی از تابع ارزش  حالت مربوط به آن خط\nf مشی را نیز محاسبه کنند. به چنین روش‌هایی، \textit{\مهم{بازیگر-منتقد}} 
\LTRfootnote{Actor-Critic}
گفته می‌شود که \textit{\مهم{بازیگر}} 
\LTRfootnote{Actor}
اشاره به خط‌مشی آموخته شده و \textit{\مهم{منتقد}} 
\LTRfootnote{Critic}  
اشاره به تابع ارزش آموخته شده (معمولا یک تابع ارزشِ حالت) دارد.

با کم کردن یک تابع دلخواه روی حالت‌ها، 
$b_t(s)$
از 
$G_t$
در طرف راست رابطه 
\ref{eq:pg}،
می\nf توان به تعمیمی از قضیه 
\ref{[th:pg]}
دست یافت
\begin{align}
	\nabla J(\theta) \propto \mathbb{E}_\pi \left[ (G_t-b(s)) \frac{\nabla_\theta \pi (A_t|S_t; \theta)}{\pi(A_t|S_t); theta} \right]
	\label{eq:pgbase}
\end{align}
برای اثبات درستی رابطه
\ref{eq:pgbase}
کافیست نشان دهیم
$\mathbb{E}_\pi \left[b(s)\right] \frac{\nabla_\theta \pi (A_t|S_t; \theta)}{\pi(A_t|S_t); theta} = 0$
واریانس این تخمین را کاهش داد به‌طوری‌که نااریب باقی بماند. به چنین تابعی \مهم{پایه} گفته می‌شود. 
در نتیجه گرادیان به شکل
$\nabla_\theta \ log \ \pi(a_t|s_t;\theta) (R_t - b_t(s_t))$
خواهد بود. معمولا از تخمینی آموخته‌شده از تابع ارزش به عنوان پایه استفاده می‌شود،
$b_t(s_t) \approx v_\pi (s_t)$،
که منجر به تخمینی با واریانس بسیار کوچک‌تر از گرادیان خط‌مشی می‌‌شود؛ در حالی‌که تخمین نااریب باقی می‌ماند و درنتیجه عملیات یادگیری با سرعت بیشتری انجام می‌شود.
این روش می‌تواند به شکل معماری \مهم{بازیگر-منتقد}  تعبیر شود که خط‌مشی $\pi$ \مهم{بازیگر} و پایه $b_t$ \مهم{منتقد} است. اگر از تخمین یک تابع ارزشِ حالت به عنوان پایه استفاده کنیم 
$b_t(s) = v_{\pi_{\theta_t}}(s)$ عبارت $R_t - b_t(s)$ می‌تواند به شکل تخمینی از \مهم{مزیت} \LTRfootnote{Advantage} عمل $a_t$ در حالت $s_t$ یا 
$A(a_t,s_t)=Q(a_t,s_t)-v(s_t)$ تعبیر شود. چراکه $R_t$ تخمینی از  $Q_\pi (a_t, s_t)$ و $b_t$ تخمینی از  $v_\pi (s_t)$ است. در این صورت به این روش 
\lr{Advantage Actor-Critic}
یا A2C گفته می‌شود.
\شروع{الگوریتم}{الگوریتم $Advantage Actor-Critic$ }
\ورودی{یک پارامتری‌شده مشتق پذیر از خط‌مشی 
	$\pi_\theta(a|s)$}
\ورودی{یک پارامتری‌شده مشتق‌پذیر از تابع ارزشِ حالت 
	$v_\omega(s)$}
\دستور{پارامترهای خط‌مشی 
	$\theta \in \mathbb{R}^{d'}$
	و تابع ارزشِ حالت
	$\omega \in \mathbb{R}^d$
	را مقداردهی اولیه کن
}
\به‌ازای{تکرار کن}
%\اگر{$|E| > 0$}
%	\دستور{یک کاری انجام بده}
%\پایان‌اگر
\دستور{حالت اولیه $S$ را بساز}
\دستور{$1 \longrightarrow I$}
\تاوقتی{$S$ حالت نهایی نیست}
\دستور{
	$A \sim \pi_\theta(.|S)$
}
\دستور{عمل $A$ را انجام بده و حالت $S'$ و پاداش $R$ را مشاهده کن}

\دستور{
	$ \longrightarrow \delta$
	$R + \gamma v_\omega(S') - v_\omega(S)$
}
\دستور{
	$ \longrightarrow \omega$
	$\omega + \alpha^\omega I \delta \nabla_\omega v_\omega(S)$
}
\دستور{
	$ \longrightarrow \theta$
	$\theta +  \alpha^\theta   I  \delta \nabla_\theta \ ln \ \pi_\theta(A| S)$
}

\دستور{
	$\gamma I \longrightarrow I$
}

\دستور{
	$S' \longrightarrow S$
}
\پایان‌تاوقتی
\پایان‌به‌ازای
%\caption{ برگرفته شده از !!!!!!!!!!!!}
\پایان{الگوریتم}
\subsubsection{روش TRPO}
%\قسمت{روش TRPO}
\شروع{تعریف}
اگر
$\rho_\pi : S \longrightarrow \mathbb{R}$
تابع فرکانس تخفیف‌دار دیده شدن حالت‌ها باشد. یعنی $$\rho_\pi(s) = P(S_0=s) + \gamma P(S_1=s) + \gamma^2 P(S_2=s) + ...$$ \\که دنباله $S_0, S_1, S_2$ خط‌مشی $\pi$ را دنبال می‌کند.
\پایان{تعریف}
اگر $\pi$ و $\pi'$ دو خط‌مشی باشند و  $J(\pi) = \mathbb{E}_\pi[R_0]$  می‌توان نشان داد 
$$J(\pi')= J(\pi) + \sum_{s} \rho_{\pi'}(s) \sum_{a} \pi'(a|s) A_\pi(s,a)$$ که  $A_\pi(s,a) = Q_\pi (s,a) - V_\pi(s)$ تابع مزیت عمل $a$ در حالت $s$ باشد.
وابستگی پیچیده  $\rho_\pi'(s)$ در طرف راست تساوی به $\pi'$ بهینه‌سازی مستقیم را مشکل می‌کند.
برای حل این مشکل شولمن و همکارانش 
مقیاس عملکرد دیگری، $L_\pi(\pi')$، را معرفی می‌کند و نشان می‌دهد که اگر $\pi$ و $\pi'$ به اندازه کافی به یکدیگر نزدیک باشند، افزایش  $L_\pi(\pi')$  همواره با افزایش $J$ همراه خواهد‌بود \cite{schulman2015trust}.
\begin{align}
	L_\pi(\pi')= J(\pi) + \sum_{s} \rho_{\pi}(s) \sum_{a} \pi'(a|s) A_\pi(s,a)
	\label{l_pi}
\end{align}
در عبارت 
\ref{l_pi}
$L_\pi$ از تابع فرکانس $\rho_\pi$ به جای $
\rho_{\pi'}$
استفاده می‌کند.

\شروع{قضیه}
فرض کنید 
$\alpha = D_{TV}^max(\pi_{old}, \pi_{new})$
باشد که 
$$D_{TV}^max (\pi, \pi') = \max_{s} D_{TV}(\pi(.|s) || \pi'(.|s)$$
و 
$D_{TV}(p || q)$
دیورژانس 
\lr{total variation}
بین دو بردار $p$ و $q$ باشد
$$D_{TV}(p || q) = \dfrac{1}{2} \sum_{i} |p_i - q_i|$$ در این صورت
$$J(\pi_{new}) \ge L_{\pi_{old}}(\pi_{new}) - \dfrac{4 \epsilon \gamma}{(1- \gamma)^2}$$ که $\epsilon = \max_{s,a} |A_\pi(s,a)|$.
\label{q:ghas}
\پایان{قضیه}
با توجه به قضیه \ref{q:ghas} و نامعادله $D_{TV}(p || q)^2 \le D_{KL}(p || q)$  که  $D_{KL} (p || q)$ برابر با دیورژانس KL دو بردار $p$ و $q$ است\cite{schulman2015trust}. می‌توان  نتیجه گرفت
\begin{align*}
	J(\pi') \ge L_{\pi}(\pi') - C \ D_{KL}^max(\pi, \pi')
\end{align*}
که
\begin{align}
	C = \dfrac{4 \epsilon \gamma}{(1-\gamma)^2}
	\label{eq:cc}
\end{align}
رابطه \ref{eq:cc} نشان می‌دهد که می‌توان یک دنباله صعودی از خط‌مشی‌ها داشت به‌طوری‌که
$$J(\pi_0) \le J(\pi_1) \le J(\pi_2) \le ... $$
چرا که فرض کنید
$M_i(\pi) = L_{\pi_i}(\pi) - C \ D_{KL}^max(\pi_i, \pi)$
در این صورت $$J(\pi_{i+1}) \ge M_i(\pi{i+1}) $$ 
$$J(\pi_i) = M_i(\pi_i)$$
بنابراین
$$J(\pi_{i+1}) - J(\pi_i) \ge M_i(\pi_{i+1}) - M_i(\pi_i)$$
می‌توان نتیجه گرفت با بیشینه کردن $M_i$ در هر گام می‌توان اطمینان حاصل کرد که مقیاس عملکرد واقعی $J$ غیرنزولی خواهد بود.
\شروع{الگوریتم}{الگوریتم $Policy Iteration$ با مقیاس عملکرد $L_\pi$}

\دستور{خط‌مشی $\pi_0$
	را مقداردهی اولیه کن
}
\به‌ازای{برای 
	$i=0,1,...$
	تکرار کن}
%\اگر{$|E| > 0$}
%	\دستور{یک کاری انجام بده}
%\پایان‌اگر
\دستور{همه مزیت‌های 
	$A_{\pi_i}(s,a)$
	را محاسبه کن
}
\دستور{
	$arg \max_{\pi} [L_{\pi_i}(\pi) - C \ D_{KL}^max(\pi_i, \pi)] \longrightarrow \pi_{i+1} = $
	که 
	$$C = (4 \epsilon \gamma) / (1-\gamma)^2$$ و  $$L_{\pi_i}(\pi) = J(\pi_i) + \sum_{s} \rho_{\pi_i}(s) \sum_{a} \pi(a|s) A_{\pi_i} (s,a)$$
}
\پایان‌به‌ازای
\پایان{الگوریتم}
اگر $\hat{A}_t$ تخمین مزیت 
$A_{\pi_t}(S_t, A_t)$
باشد که در گام $t$ محاسبه می‌شود، می‌توان نشان داد که در روش
\lr{TRPO}
مقیاس عملکرد $L_\pi$ در هر گام به شکل؛
$$\mathbb{E}_t\left[\dfrac{\pi_{\theta_{new}}(A_t| S_t)}{\pi_{\theta_{old}}(A_t|S_t)} \hat{A}_t \right]$$خواهد بود.
\subsubsection{روش PPO}
%\قسمت{روش PPO}
در روش \lr{TRPO} دیدیم که بیشینه‌سازی مقیاس عمکلرد $L_\pi$ ساده‌تر از مقیاس عملکرد $J$ است ولی در عوض الگوریتم صعود گرادیان تنها مجاز به اعمال تغییرات کوچک در خط‌مشی است. یک راه دیگر برای کنترل تغییرات خط‌مشی استفاده از تابع CLIP است. در روش
\lr{TRPO}
دیدیم که تابع مقیاس عملکرد، در گام $t$ به شکل زیر است؛
$$L_{\pi_{old}}(\pi_{new}) = \mathbb{E_t}\left[\dfrac{\pi_{\theta_{new}}(A_t| S_t)}{\pi_{\theta_{old}}(A_t|S_t)} A_{\pi}(S_t, A_t)\right]$$
فرض کنید 
$r_t(\theta_{new})$
نسبت احتمالات 
$r_t(\theta_{new}) = \dfrac{\pi_{\theta_{new}}(a_t|s_t)}{\pi_{\theta_{old}}(a_t|s_t)}$ باشد. بنابراین
$$L_{\pi_{old}}(\pi_{new}) = \mathbb{E}_t\left[\dfrac{\pi_{\theta_{new}}(A_t| S_t)}{\pi_{\theta_{old}}(A_t|S_t)} A_{\pi}(S_t, A_t)\right] = \mathbb{E}_t\left[r_t(\theta_{new}) A_{\pi_{old}}(S_t,A_t)\right].$$ مقیاس عملکرد  $L^{CLIP}(\theta)$ را به شکل زیر تعریف می‌کنیم
$$L^{CLIP}(\theta) = \mathbb{E}_t\left[min(r_t(\theta) \hat{A}_t, clip(r_t(\theta), 1-\epsilon, 1+\epsilon) \hat{A}_t \right]$$
که $\epsilon$ یک ابرپارامتر \LTRfootnote{Hyperparameter} مثلا 
$\epsilon=0.2$
است. اولین عبارت داخل min همان مقیاس عملکرد روش \lr{TRPO} است. در عبارت دوم مقادیر بزرگتر از $1+\epsilon$ یا کوچکتر از 
$1-\epsilon$
در $r_t(\theta)$ به ترتیب به $1+\epsilon$ و $1-\epsilon$ تغییر پیدا کرده‌اند تا تغییرات بزرگ خط‌مشی را کنترل کنند. نهایتا عبارت کوچکتر از میان این دو انتخاب خواهد شد. بنابراین اگر مقیاس عملکرد clip نشده (عبارت اول) کوچکتر یا مساوی با حالت clip شده (عبارت دوم) باشد، $L^{CLIP}$ دقیقا همان مقیاس عملکرد $L_\pi$ خواهد بود. در غیر این صورت مقیاس عملکرد clip شده انتخاب می‌شود تا از تغییرات بزرگ خط‌مشی جلوگیری شود.

\section{روش‌های مبتنی بر ارزش}
%\قسمت{روش‌های مبتنی بر ارزش}


در روش‌های یادگیریِ تقویتی بدون مدل مبتنی بر ارزش
تابع ارزش عمل با استفاده از  یک تخمین‌گر تابع
\LTRfootnote{Function approximator}
 ، مانند شبکه عصبی ، نشان داده می‌شود. فرض کنید
$Q(s,a;\theta)$
یک تابع ارزش عمل تقریبی با پارامتر  
$\theta$ 
باشد.
الگوریتم‌های مختلفی برای به‌روزرسانی $\theta$ وجود دارد.
الگوریتم $ Q-learning$ یکی از نمونه‌های چنین الگوریتمی‌است
که هدف آن تقریب مستقیم تابع ارزشِ عمل بهینه 
$Q^*(s,a) \approx Q(s,a: \theta)$
 است. در $ Q-learning$ یک مرحله‌ای، پارامترهای $\theta$ از تابع ارزشِ عمل با به حداقل رساندن تابع هزینه به شکل مرحله به مرحله آموخته می‌شوند، به شکلی که تابع هزینه $i$ام به شکل 
$$L_i(\theta_i) = \mathbb{E} {\left( r+\gamma \max_{a'} Q(s',a'; \theta_{i-1})- Q(s,a:\theta_i) \right)}^2$$
 تعریف می‌شود که 
 $s'$
 حالتی است که بعد از حالت $s$ دیده می‌شود.
 \subsubsection{روش‌های Q-learning}
%\قسمت{روش‌های Q-learning}

خانواده روش‌های  Q-learning تلاش می‌کنند مستقیما تابع ارزش عمل-حالت بهینه $Q^*(s,a)$ را تخمین بزنند. آن‌ها به طور معمول از یک تابع هدف مبتنی بر معادله بلمن استفاده می‌کنند. این بهینه‌سازی تقریباً همیشه به صورت off-policy انجام می‌شود، به این معنی که هر به‌روزرسانی می‌تواند از داده‌های جمع‌آوری شده در هر نقطه استفاده کند، بدون اینکه در نظر بگیرد نحوه انتخاب عامل برای کشف محیط در هنگام به‌دست آوردن داده‌ها چگونه بوده است. خط‌مشی مربوطه از طریق ارتباط بین 
$Q^*$
و
$\pi^*$
 به‌دست می‌آید. 
 عامل بعد از یادگرفتن تابع $Q_\theta(s,a)$ به طوری‌که  $Q_\theta(s,a) \approx Q^*(s,a)$ می‌تواند عمل بهینه در حالت $s$ را به به صورت زیر محاسبه کند $$a(s) = arg \max_a Q_{\theta}(s,a).$$
از جمله الگوریتم‌های Q-learning می‌توان به موارد زیر اشاره کرد:
\begin{itemize}
\item روش کلاسیک DQN که حوزه یادگیری تقویتی ژرف\LTRfootnote{Deep reinforcement learning} را عمیقا ارتقا بخشید.
\item روش C51 که توزیعی روی عایدی را می‌آموزد که امیدریاضی آن $Q^*$ است.
\end{itemize}
\subsubsection{روش DQN}
روش DQN 
\ref{andrychowicz2017hindsight}
 یک الگوریتم یادگیری تقویتی بدون مدل برای فضای عمل گسسته است. در این روش، تخمینی از تابع ارزش عمل بهینه $Q^*$ محاسبه می\nf شود. یک شبکه عصبی، که 
\textit{\مهم{شبکه Q}}\LTRfootnote{Q-network}
  نامیده می\nf شود، به عنوان تقریب\nf گر تابع $Q^*$ استفاده می\nf شود.
در از روش، که یک روش مستقل از خط\nf مشی است، دوره\nf ها توسط یک خط\nf مشی 
\epsilon حریصانه
مربوط به تقریب فعلی تابع ارزش، $Q$، تولید می\nf شود. چهارتایی\nf های انتقال،
$s_t, a_t, r_t, s_{t+1}$
که در طول آموزش به وجود می\nf آیند، در جایی به نام
\textit{انبار تکرار}\LTRfootnote{Replay Buffer}
ذخیره می\nf شوند. شبکه Q توسط الگوریتم نزول گرادیان روی تابع هزینه 
$L_i$
که تابعی از پارامترهای شبکه، $\theta_i$ است، روی یک نمونه تصادفی از تجربیات درون انبار تکرار، آموزش داده می\nf شود.

یک چالش جدی که در استفاده از شبکه عصبی برای یادگیری تقویتی وجود دارد این است که اغلب الگوریتم های بهینه سازی، فرض می کنند که نمونه ها به طور یکنواخت و مستقل از یکدیگر، از فضای داده ها انتخاب می شوند. درحالیکه در یادگیری تقویتی، این فرض معمولا برقرار نیست؛ به خصوص اگر نمونه ها از طریق کاوش در محیط واقعی به دست آمده باشد. یکی از روش های برای حل این مشکل، استفاده از انبار تجربه است. در هر گام، تجربه به دست آمده را در حافظه ای به نام انبار تجربه ذخیره می کنیم. هر تجربه به شکل چهارتایی $(s,a,r,s')$ ذخیره می شود. از انبار تجربه می توان برای نمونه گیری و دسته \LTRfootnote{batch} سازی استفاده کرد. اگر انبار تجربه پر شود، کهنه ترین تجربیات، حذف می شوند تا تجربیات تازه جایگزین آنها شود

برای پایداری بیشتر روند بهینه سازی، $y_t$ با استفاده از 

πQ (s) = argmaxa∈AQ (s ، a). سیاست گرانبها w.r.t. Q سیاستی است که احتمال آن اتخاذ می شود
یک عمل تصادفی (به طور یکنواخت از A نمونه گیری شده) و عمل πQ (ها) را با احتمال 1 - انجام می دهد.
در طول آموزش ، ما قسمتهایی را با استفاده از سیاست -بخوردگی w.r.t ایجاد می کنیم. تقریب فعلی از
تابع عمل-مقدار Q. نوکل های انتقالی (st ، at ، rt ، st + 1) که در طول آموزش دیده می شوند ، هستند
در بافر پخش به اصطلاح ذخیره می شود. نسل اپیزودهای جدید با ترکیب عصبی ترکیب می شود
آموزش شبکه. این شبکه با استفاده از شیب شیب مینی دسته ای در از دست دادن L که آموزش داده می شود
عملکرد Q تقریبی را برای جلب رضایت معادله بلمن تشویق می کند: L = E (Q (st، at) - yt)
2
،
که در آن yt = rt + γ maxa0∈A Q (st + 1، a0
) و از tuples (st، at، rt، st + 1) از پخش نمونه برداری می شود
بافر 1
.
به منظور ثبات بیشتر این روش بهینه سازی ، معمولاً اهداف با استفاده از a محاسبه می شوند
شبکه هدف جداگانه که با سرعت کمتری نسبت به شبکه اصلی تغییر می کند. یک عمل معمول
1 اهداف به پارامترهای شبکه بستگی دارد اما این وابستگی در طی انتشار مجدد نادیده گرفته می شود.
2
برای تنظیم دوره ای وزن شبکه هدف بر وزن فعلی شبکه اصلی است (به عنوان مثال
منیه و همکاران (2015)) یا برای استفاده از میانگین polyak2
(Polyak و Juditsky ، 1992) نسخه اصلی
در عوض شبکه (Lillicrap و همکاران ، 2015).

Deep Q-Networks (DQN) (Mnih et al., 2015) is a model-free RL algorithm for discrete action
spaces. Here we sketch it only informally, see Mnih et al. (2015) for more details. In DQN we
maintain a neural network Q which approximates Q∗
. A greedy policy w.r.t. Q is defined as
πQ(s) = argmaxa∈AQ(s, a). An -greedy policy w.r.t. Q is a policy which with probability  takes
a random action (sampled uniformly from A) and takes the action πQ(s) with probability 1 − .
During training we generate episodes using -greedy policy w.r.t. the current approximation of
the action-value function Q. The transition tuples (st, at, rt, st+1) encountered during training are
stored in the so-called replay buffer. The generation of new episodes is interleaved with neural
network training. The network is trained using mini-batch gradient descent on the loss L which
encourages the approximated Q-function to satisfy the Bellman equation: L = E (Q(st, at) − yt)
2
,
where yt = rt + γ maxa0∈A Q(st+1, a0
) and the tuples (st, at, rt, st+1) are sampled from the replay
buffer1
.
In order to make this optimization procedure more stable the targets yt are usually computed using a
separate target network which changes at a slower pace than the main network. A common practice
1The targets yt depend on the network parameters but this dependency is ignored during backpropagation.
2
is to periodically set the weights of the target network to the current weights of the main network (e.g.
Mnih et al. (2015)) or to use a polyak-averaged2
(Polyak and Juditsky, 1992) version of the main
network instead (Lillicrap et al., 2015).
%\قسمت*{روش DQN}
معمولا برای تقریب زدن توابع ارزش در یادگیری تقویتی، از یک تابع خطی استفاده می‌شود.
اما گاهی اوقات از یک تقریب عملکرد غیرخطی به جای آن، مانند یک شبکه عصبی هم می‌توان استفاده کرد. شبکه‌های عصبی با عنوان شبکه Q\LTRfootnote{Q-Network} شناخته می‌شوند.
شبکه Q را می‌توان با کمینه ساختن دنباله‌ای از توابع هزینه به شکل 
$L_1(\theta_1), L_2(\theta_2), L_3(\theta_3), ... $ آموزش داد؛ به طوری‌که
$$L_i(\theta_i)=\mathbb{E}\left[(y_i - Q(s,a;\theta_i))^2\right]$$  و  $$y_i = \mathbb{E}[r + \gamma \max_{a'} Q(s',a'; \theta_{i-1})| s,a].$$ با مشتق گرفتن از تابع هزینه نسبت به پارامترهای $\theta_i$  خواهیم داشت: $$\nabla_{\theta_i} L_i{\theta_i} = \mathbb{E}\left[ \left(r + \gamma \max{a'} Q(s',a';\theta_{i-1}) - Q(s,a;\theta_i)\right) \nabla_{\theta_i} Q(s,a;\theta_i)\right].$$
%Rather than computing the full expectations in the above gradient, it is often computationally expedient to optimise the loss function by stochastic gradient descent. If the weights are updated after
%every time-step, and the expectations are replaced by single samples from the behaviour distribution
%ρ and the emulator E respectively, then we arrive at the familiar Q-learning algorithm [26].
به جای محاسبه امیدریاضی کامل در گرادیان فوق، غالباً از نظر محاسباتی، بهینه‌سازی تابع هزینه با نزول گرادیان تصادفی  \LTRfootnote{stochastic gradient descend} راه‌حل بهتری است. اگر در هر مقطع زمانی، وزن‌ها به روزرسانی شود و امیدریاضی با یک نمونه از توزیع خط‌مشی رفتار \LTRfootnote{behavior policy} جایگزین شود، الگوریتم
\ref{alg:qlearnn}
 Q-learning را نشان می‌دهد.
\شروع{الگوریتم}
{الگوریتم Q-learning با Experience replay}

\دستور{حافظه 
	replay 
	$D$
را مقدار دهی اولیه کن}
\دستور{تابع ارزشِ عمل Q را با وزن‌های تصادفی مقداردهی اولیه کن}
\به‌ازای{برای هر اپیزود 
$1...M$}
\دستور{دنباله 
	$d_1 = \{S_1\}$
	 و کدینگ 
	 $\phi_1 = \phi(d_1)$
	  را مقداردهی اولیه کن}
\به‌ازای{برای $t=1...T$}
\دستور{با احتمال $\epsilon$ یک عمل تصادفی $a_t$ را انتخاب کن، در غیر این صورت 
	$a_t = \max_{a} Q^*(\phi(d_t),a;\theta)$
	 را انتخاب کن}
 \دستور{عمل $a_t$ را انجام بده و حالت $S_{t+1}$ و پاداش $R_t$ را مشاهده کن}
 \دستور{قرار بده 
 	$d_{t+1} = d_t,a_t,S_{t+1}$
 	 و 
 	 $\phi_{t+1} = \phi(d_{t+1})$
  }
\دستور{تجربه 
	$(\phi_t, A_t, R_t, \phi_{t+1})$
	 را در $D$ ذخیره کن}
 \دستور{یک نمونه تصادفی از تجریه‌های  
 	$(\phi(j), A_j, R_j, \phi_{j+1})$
 	از انبار تجربیات $D$ انتخاب کن}
 \دستور{قرار بده 
 	\lr{
 	$y_j =$ 
 	\begin{cases}
 		$r_j$
 		 &
 		  $\phi_{j+1} \  terminal$ \\
 		$r_j$ & $otherwise$
 	\end{cases}}
}
\دستور{یک گام از نزول گرادیان را برای تابع هزینه 
	$(y_j - Q(\phi_j, a_j; \theta))^2$
	 انجام بده}
  \پایان‌به‌ازای
\پایان‌به‌ازای
\label{alg:qlearnn}
\پایان{الگوریتم}
توجه داشته باشید که الگوریتم \ref{alg:qlearnn} یک الگوریتم بدون مدل است. این کار وظیفه یادگیری تقویتی را مستقیماً با استفاده از نمونه‌های شبیه ساز E بدون ساختن صریح تخمین E حل می‌کند.
\\ استراتژی حریصانه، 
 $a = \max_{a} Q(s, a; \theta)$
 ، در حالی‌که یاد می‌گیرد که
کاوش کافی در فضای حالت را تضمین کند. توزیع رفتار اغلب توسط یک استراتژی Greed انتخاب می‌شود که استراتژی حریصانه را با احتمال 1 دنبال می‌کند و یک
اقدام تصادفی با احتمال $\epsilon$
!!!!!!!!!!!!!!
\subsubsection{روش C51 }
\begin{align}
Q^*(x,a)= \mathbb{E} R(x,a)+\gamma \mathbb{E}_P \max_{a' \in \EuScript{A}} Q^* (x',a').
\label{eq:Q^*}
\end{align}
در معادله \ref{eq:Q^*} $Q^*$ نقطه ثابت منحصر به فرد است، تابع ارزش بهینه، که مربوط است به مجموعه خط‌مشی‌های بهینه 
$\Pi^*$
($\pi*$ بهینه است اگر
$\mathbb{E}_{a\sim \pi^*} Q^*(x,a) = \max_{a}Q^*(x,a).$)

تابع ارزش و تابع پاداش چشمداشتی را بردارهایی در فضای 
$\mathbb{R}^{\EuScript{X \times A}}$ 
در نظر میگیریم. در این حالت عملگر بلمن، $\EuScript{T}^\pi$، و عملگر بهینه $\EuScript{T}$ به شکل زیر است:
\begin{align}
\EuScript{T}^\pi Q(x,a) := \mathbb{E} R(x,a) + \gamma \mathbb{E}_{P,\pi} Q(x',a') \\
\EuScript{T} Q(x,a) := \mathbb{E} R(x,a) + \gamma \mathbb{E}_P \max_{a' \in \EuScript{A}} Q(x',a')
\end{align}
عملگر بلمن و عملگر بهینه، به طور خاص، هردو نگاشتی انقباضی هستند و اعمال مکرر آنها روی یک $Q_0$ اولیه، به صورت نمایی به ترتیب به $Q^\pi$ و $Q^*$ همگرا می‌شود
\cite{bertsekas1996neuro}.
 امید ریاضی از معادله بلمن بیرون کشیده می‌شود و در عوض یک توزیع کامل از متغیر تصادفی 
$Z^\pi$ 
قرار داده می‌شود که نگاشتی از دوتایی حالت عمل به توزیعی از عایدی‌ها است. تابع 
$Z^\pi$ 
 توزیع ارزش می‌‌باشد.
\شروع{تعریف}
برای 
$F$
 و 
 $G$
که دو تابع توزیع تجمعی بر روی اعداد حقیقی هستند، تعریف می‌شود.
\begin{align}
d_p (F,G):= \inf_{U,V} \parallel U-V \parallel_{p},
\label{eq:inf}
\end{align}
که اینفیمم روی تمام جفت متغیرهای تصادفی  
 $(U,V)$
  که تابع توزیع تجمعی مربوط به آن‌ها به ترتیب 
 $F$
 و 
 $G$
 باشد، گرفته می‌شود.
 اینفیمم توسط تبدیل معکوس تابع توزیع تجمعی روی یک متغیر تصادفی 
 $\EuScript{u}$
 با توزیع یکنواخت در بازه 
 $[0,1]$
 حاصل می‌شود:
 $$d_p (F,G) = \parallel F^{-1}(\EuScript{U})-G^{-1} (\EuScript{U})\parallel_{p}$$
 برای 
$p < \infty$
می‌توان نوشت:
$$d_p (F,G) = \left( \int_{0}^{1} \mid F^{-1}(u)-G^{-1} (u)\mid^{p} du \right)^{1/p}$$
با داشتن دو متغیر تصادفی 
$(U,V)$
با توابع توزیع تجمعی 
$F_U,F_V$
می‌توان نوشت
$$d_p(U,V) := d_p(F_U,F_V)$$
 می‌توان با قرار دادن خود متغیر‌های
   $U$ و $V$
     به جای توابع توزیع تجمعی‌شان در فرمول 
     \ref{eq:inf}
      به طور ساده تر نوشت:
$$d_p(U,V)= \inf_{U,V} \parallel U-V\parallel_p$$ 
\پایان{تعریف}

\شروع{تعریف}
\EuScript{Z} 
فضای توزیع ارزش با ممان‌های
\LTRfootnote{Moment}
 محدود تعریف می‌شود.
برای دو توزیع ارزش 
$Z_1$
و
$Z_2$
عضو 
\EuScript{Z} 
از فرم ماکسیمال متریک واسرشتاین
\LTRfootnote{Wasserstein}
استفاده می‌کنیم:
$$\bar{d_p}(Z_1,Z_2) := \sup_{x,a} d_p(Z_1(x,a),Z_2(x,a))$$
از 
$\bar{d_p}$
برای همگرایی عملگرهای توزیعی بلمن
\LTRfootnote{Distributional Bellman Operators}
 استفاده می‌شود.
\پایان{تعریف}
\paragraph{ارزیابی خط‌مشی}
در روند ارزیابی خط‌مشی، مطلوب است تابع ارزش
$V^\pi$
مرتبط با خط‌مشیِ $\pi$ داده شده. 
تابع پاداش به عنوان یک بردار تصادفی، 
$R \in \EuScript{Z}$،
وعملگر انتقال تعریف می‌شود،
$P^\pi : \EuScript{Z} \rightarrow \EuScript{Z}$
\begin{align}
P^{\pi} Z(x,a) := Z(X',A') \\ \numberthis
X' \sim P(.|x,a), A'\sim \pi (.|X'), \nonumber
\end{align}
عملگرتوزیعی بلمن،
$\EuScript{T}^\pi : \EuScript{Z} \rightarrow \EuScript{Z}$،
 در نظر گرفته می‌شود:
\begin{align}
 \EuScript{T}^\pi Z(x,a) := R(x,a)+ \gamma P^\pi Z(x,a)
\end{align}  
می‌توان نشان داد که $T^\pi$ یک نگاشت انقباضی 
\LTRfootnote{Contraction Mapping}
روی
 $Z$
  است که نقطه ثابت آن توزیع ارزش 
$Z^\pi$
    است.
اگر 
$\Pi^*$
مجموعه‌ای از خط‌مشی‌های بهینه باشد. 
تابع توزیع ارزش بهینه تعریف میشود:
\شروع{تعریف}
توزیع ارزش بهینه، تابع توزیعِ ارزش خط‌مشیِ بهینه است. مجموعه توزیع ارزش‌های بهینه تعریف می‌شوند:
$$\EuScript{Z}^* := {Z^{\pi^{*}} : \pi^* \in \Pi^* }$$
تاکید می‌شود که همه توزیع ارزش‌ها با مقدار چشمداشتی $Q^*$ 
بهینه نیستند؛ تنها در صورتی بهینه هستند که مطابق با توزیع کامل عایدی بر‌اساس خط‌مشی بهینه باشند.
\پایان{تعریف} 
\شروع{تعریف}
خط‌مشی حریصانه
$\pi$
برای
$Z \in \EuScript{Z}$
مقدار چشمداشتی
$Z$
 را بیشینه می‌کند.
 مجموعه خط‌مشی‌های حریصانه برای 
 $Z$
 به این شکل تعریف می‌شوند:
 $$\EuScript{G}_Z := \lbrace \pi : \sum_{a} \pi (a|x) \ \mathbb{E} \ Z(x,a) = \max_{a' \in \EuScript{A}} \ \mathbb{E} \ Z(x,a') \rbrace.$$
\پایان{تعریف}

\شروع{تعریف}
یک توزیع ارزش بهینه ناپایدار،
$Z^{**}$،
توزیع ارزشی‌ است که مربوط به دنباله خط‌مشی‌های بهینه است. به مجموعه‌ای از توزیع ارزش‌های بهینه ناپایدار 
 $\EuScript{Z}^{**}$
 گفته می‌شود.
\پایان{تعریف}

\شروع{قضیه}
فرض می‌شود 
$\EuScript{X}$
قابل اندازه‌گیری و 
$\EuScript{A}$
محدود باشد.
بنابراین:
$$\lim_{k\to\infty} \inf_{Z^{**} \in \EuScript{Z}^{**}} d_p (Z_k (x,a), Z^{**}(x,a)) = 0  \ \forall x,a.$$
\پایان{قضیه}

\subsubsection{ روش SAC}
نشان داده می‌شود که می‌توان الگوریتم بازیگر-منتقد بیشینه آنتروپی مستقل از خط مشی، که بازیگر-منتقد نرم
\LTRfootnote{Soft Actor Critic(SAC)}
  نامیده‌ می‌شود را طراحی کرد.
این الگوریتم به سادگی به کارهای بسیار پیچیده با ابعاد بالا بسط داده می‌شود، جایی که روش‌های مستقل از خط مشی‌ مانند
 $DDPG$
  معمولا برای رسیدن به نتایج مناسب، با چالش روبه‌رو هستند.
  
یادگیری خط‌مشی در فضاهای عمل پیوسته آدرس‌دهی می‌شود. یک فرایند تصمیم‌گیری مارکوف افق-بی‌نهایت
\LTRfootnote{Infinite Horizon Markov Decision}
 در نظر گرفته می‌شود که در آن فضای حالت
  $S$
   و فضای عمل
   $A$
    پیوسته هستند.
یک مقیاس عملکرد عمومی‌تر بیشینه آنتروپی بررسی خواهد شد.
به عنوان مثال ببینید: 
!!!!!!!!!!!!!!!!
که با تقویت مقیاس عملکرد آنتروپی چشم‌داشتی خط‌مشی به نفع خط‌مشی‌های تصادفی عمل می‌کند.
\begin{align}
J_{(\pi)}= \sum\limits_{t=0}^T \mathbb{E}_\pi \left[ r(s_t , a_t) + \alpha \EuScript{H} (\pi(.|s_t)) \right]
\label{eq:jpi}
\end{align}
می‌توان الگوریتم بازیگر-منتقد نرم مستقل از خط‌مشی را با شروع از نوعی دیگر از متد تکرار خط‌مشیِ بیشینه آنتروپی استخراج کرد.
با به‌دست آوردن تکرار خط‌مشی نرم شروع خواهد شد، که یک الگوریتم عمومی برای یادگیری خط‌مشی‌های بیشینه آنتروپی بهینه که در چارچوب بیشینه آنتروپی بین ارزیابی و بهبود خط‌مشی در حال تناوب است.
در گام ارزیابی خط‌مشی امید است مقدار خط‌مشی
$\pi$
 براساس مقیاس عملکرد بیشینه آنتروپی در معادله \ref{eq:jpi} محاسبه شود. برای یک خط‌مشی ثابت، مقدار $Q$  نرم می‌تواند به صورت گام‌به‌گام با شروع از هر تابع
  $Q:S \times R \rightarrow R$ 
 و اعمال مکرر یک عملگر پشتیبان بلمن اصلاح شده،
 $\EuScript{T}^\pi$
  که در تابع ارزش حالت نرم به صورت زیر داده شده است، محاسبه شود:
\begin{align}
\EuScript{T}^\pi Q(s_t,a_t) \triangleq r(s_t,a_t)+\gamma \mathbb{E} \left[ V(s_t+1) \right]
\label{eq:tpi}
\end{align}
\begin{align}
V(s_t)= \mathbb{E}_\pi \left[ Q(s_t,a_t)- \log \pi(a_t | s_t) \right]
\end{align}
می‌توان تابع ارزش نرم برای هر خط‌مشی$\pi$ را با اعمال مکرر، $T^\pi$ به‌دست آورد.
\begin{align}
\pi_{new} = arg \min_{\pi' \in \Pi} D_{KL}  \left( \pi'(.|s_t) \parallel \frac{\exp{(Q^{\pi_{old}} (s_t,.))}}{Z^{\pi_{old}} (s_t)}\right)
\label{eq:pinew}
\end{align}
تابع پارش 
$Z^\pi (s_t)$
توزیع را یکنواخت می‌کند و از آنجایی که در حالت عمومی غیرقابل حل است، به گرادیان نسبت به خط‌مشی جدید کمکی نمی‌کند و به همین دلیل می‌تواند نادیده گرفته شود. برای این تابع می‌توان نشان داد که خط‌مشی جدید نسبت به خط‌مشی قدیمی با توجه به مقیاس عملکرد معادله‌ی \ref{eq:jpi} مقدار بیشتری دارد. 
الگوریتم تکرار خط‌مشی نرم کامل، بین ارزیابی خط‌مشی نرم و بهبود خط‌مشی نرم در تناوب است و در میان خط‌مشی‌های عضو
$\Pi$،
احتمالا به خط‌مشی بیشینه آنتروپی بهینه همگرا می‌شود.
 \paragraph{بازیگر-منتقد نرم}
به‌دست آوردن یک تقریب تجربی برای تکرار خط‌مشی نرم در دامنه‌های پیوسته‌ی بزرگ موردنیاز است.  در انتها، تقریب‌گرهای تابع برای تابع
$ Q$
 و خط‌مشی استفاده خواهند شد به جای اجرای بهبود و ارزیابی برای همگرایی، بین بهینه‌سازی هر دو شبکه با نزول گرادیان تصادفی در تناوب است. تابع مقدار حالت پارامتری شده
 $V_\psi (s_t)$
 ، تابع$Q$ نرم 
 $Q_\theta (s_t,a_t)$
  و خط‌مشی قابل حل 
  $\pi_\phi (a_t | s_t)$
  بررسی خواهند شد. پارامترهای این شبکه‌ها 
  $\psi, \theta, \phi$
   هستند. به عنوان مثال، توابع ارزش می‌توانند به عنوان شبکه‌های عصبی گویا و خط‌مشی به عنوان یک گوسی با میانگین و همگرایی گرفته شده از شبکه‌های عصبی، مدل شوند.
تابع ارزش نرم برای کمینه کردن مربع خطای باقیمانده‌ها آموزش دیده شده‌است.
که 
$\EuScript{D}$
 توزیع حالت‌ها و عمل‌های نمونه‌گیری شده قبلی یا انبار تجربه
 \LTRfootnote{Replay Buffer}
  است.
\begin{align}
\hat{\nabla}_\psi J_V (\psi) = \nabla_\psi V_\psi (s_t) (V_\psi (s_t)-Q_\theta(s_t,a_t)+\log \pi_\phi (a_t |s_t)) ,
\end{align}
عمل‌ها به جای انبار تجربه، براساس خط‌مشی فعلی نمونه‌گیری شده‌اند. پارامترهای تابع $Q$ نرم می‌تواند برای کمینه کردن باقیمانده‌ی بلمن آموزش دیده‌شود.
\begin{align}
J_Q(\theta) = \mathbb{E}_{\EuScript{D}} \left[ \frac{1}{2} \left( Q_\theta (s_t,a_t)- \hat{Q} (s_t,a_t)\right)^2 \right] ,
\end{align}
با 
\begin{align*}
\hat{Q} (s_t,a_t) = r(s_t,a_t) + \gamma \mathbb{E} [V_\Psi (s_t+1)] ,
\end{align*}
که می‌تواند مجددا با گرادیان‌های تصادفی بهینه شود.
\begin{align}
\hat{\nabla_{\theta}} J_Q (\theta) = \nabla_\theta Q_\theta (a_t,s_t) \left( Q_\theta (s_t,a_t)- r(s_t,a_t) - \gamma V_\Psi (s_t+1) \right). 
\label{eq:hatnabla}
\end{align}
در آخر، پارامترهای خط‌مشی می‌توانند با کمینه کردن مستقیم همگرایی-KL چشمداشتی در معادله \ref{eq:pinew} آموخته شوند:
\begin{align}
J_\pi(\phi) = \mathbb{E}\left[ D_{KL}  \left( \pi_\phi(.|s_t) \parallel \frac{\exp{(Q_{\theta} (s_t,.))}}{Z_{\theta} (s_t)}\right) \right]
\end{align}
!!!!!!!!!!!!!!!!!!!!!!!!!!!!!!!!!!!!!!!!!!!!!!!!!!!!!!!!!!
\شروع{الگوریتم}{الگوریتم $Policy Iteration$ با مقیاس عملکرد $L_\pi$}
\دستور{خط‌مشی $\pi_0$
	را مقداردهی اولیه کن
}
$a_t \approx \pi_\phi(a_t|s_t)$ \\
$s_{t+1} \approx p(s_{t+1}|s_t,a_t)$\\
$\EuScript{D}\leftarrow \EuScript{D} \bigcup {(s_t,a_t,r(s_t,a_t),s_{t+1})}$ \\
$\psi \leftarrow \psi - \landa_V \hat{\nabla}_\spi J_V (\psi)$ \\
$\theta_i \leftarrow \theta_i - \landa_Q \hat{\nabla}_{\theta_i} J_Q(\theta_i) for i \in {1,2}$\\
$\phi \leftarrow \phi - \landa_\pi \hat{\nabla}_\phi J_\pi (Q)$ \\
$\bar{\psi} \leftarrow \tau \psi + (1-\tau) \psi$
\پایان{الگوریتم}
که \ref{eq:hatnabla} یک بردار نویز ورودی است که از یک توزیع ثابت مانند یک گوسی کروی نمونه‌گیری شده‌است.
\subsubsection{روش DDPG}
گسسته سازی فضای عمل یک روش برای سازگار‌کردن و انطباق روش‌های یادگیری تقویتی عمیق نظیر 
\lr{DQN}
با دامنه‌های پیوسته می‌باشد. با این حال، این روش محدودیت‌های زیادی دارد، مخصوصا مشکل نفرین ابعاد. نفرین ابعاد بیان‌گر این است که تعداد عمل‌ها به صورت نمایی با تعداد درجات آزادی افزایش پیدا می‌کند. 
در روشی جدید یک الگوریتم بازیگر-منتقد مستقل از خط‌مشی، مستقل از مدل
\LTRfootnote{A model-free, off-policy actor-critic algorithm}
با استفاده از تخمین‌گر تابع عمیق ارائه می‌دهند که می‌تواند خط‌مشی‌ها را در فضاهای عمل پیوسته با ابعاد بالا یاد بگیرد.
این روش بر اساس الگوریتم گرادیان خط‌مشی معین
\LTRfootnote{Deterministic policy gradient (DPG)}
است که آن را گرادیان خط‌مشی معین عمیق
\LTRfootnote{Deep DPG (DDPG)} 
می‌نامیم. 
%Here we combine the actor-critic approach with insights from Deep Q Network (DQN)
در آن روش، رویکرد بازیگر-منتقد را با بینش شبکه \lr{Q} عمیق
\LTRfootnote{Deep Q Network (DQN)} 
ترکیب می‌کنیم.
الگوریتم 
\lr{DPG}
تابع عمل پارامتری
$
\mu (s|\theta^\mu)	
$
را نگهداری می‌کند که با نگاشت معین حالت‌ها به یک عمل خاص، سیاست فعلی را مشخص می‌کند.
همانند روش یادگیری 
\lr{Q}
در اینجا نیز منتقد
$Q(s,a)$
با استفاده از معادله بلمن آموخته می‌شود.
%the actor is updated by following applying the chain rule to the expected return from the start distribution with respect to the actor parameters 
بازیگر با پیروی از اعمال قاعده زنجیره‌ای روی عایدی چشم‌داشتی از توزیع شروع
$J$
نسبت به پارامترهای بازیگر به‌روز می‌شود.
\begin{align}
	\nabla_{\theta \mu} J = & \mathbb{E}_{s_t \sim \rho ^{ \beta}} [ \nabla_{\theta \mu} Q(s,a|\theta^Q)|_{s=s_t,a=\mu (s_t|\theta^{\mu})}  ] \\ \nonumber
	= & \mathbb{E}_{s_t \sim \rho ^{ \beta}} [ \nabla_{\theta \mu} Q(s,a|\theta^Q)|_{s=s_t, a=\mu (s_t)} \nabla_{\theta_{\mu}} \mu (s|\theta^{\mu})|_{s=s_t} ]
\end{align}
در این روش نیز، مانند روش 
\lr{DQN}
از انبار تجربه استفاده می‌شود. 
در هر مرحله، بازیگر و منتقد با نمونه‌برداری یکنواخت از انبار به‌روزرسانی می‌شوند. از آنجا که 
\lr{DDPG}
الگوریتمی مستقل از خط‌مشی می‌باشد، انبار تجربه می‌تواند بزرگ باشد، که به الگوریتم اجازه می‌دهد تا از یادگیری مجموعه‌ای از انتقال‌های ناهمبسته بهره‌مند شود.
چالش اصلی یادگیری در فضاهای عمل پیوسته، اکتشاف است. یک مزیت  الگوریتم‌های مستقل از خط‌مشی مانند 
\lr{DDPG}
این است که می‌توان به مسئله کاوش، مستقل از الگوریتم یادگیری پرداخت.
می‌توان یک خط‌مشی اکتشاف 
$\mu'$
را با اضافه کردن نویز (نمونه‌گیری شده از فرآیند $\EuScript{N}$) به خط‌مشی بازیگر ساخت:
\begin{equation}
\mu'(s_t) = \mu(s_t|\theta_t^mu) + \EuScript{N}
\end{equation}
می‌توان $\EuScript{N}$ 
را متناسب با محیط انتخاب کرد.
\شروع{الگوریتم}{الگوریتم DDPG}
\دستور{پارامترهای 
	$\theta_\mu$
	 و 
	 $\theta_Q$
	 به ترتیب مربوط به بازیگر 
	 $\mu(s;\theta_\mu)$
	 و منتقد
	 $Q(s,a;\theta_Q)$
	 را مقداردهی اولیه کن.
}
\دستور{پارامترهای توابع هدف $\mu'$ و $Q'$ را با وزن‌های 
$\theta_{\mu'} \longleftrightarrow \theta_\mu$
و
$\theta_{Q'} \longleftarrow \theta_Q$
مقداردهی اولیه کن
}
\دستور{حافظه تکرارها $R$ را بساز}
\‌به‌ازای{برای هر اپیزود $1...M$}
\دستور{یک تابع نویز تصادفی $\mathbb{N}$} بساز
\دستور{حالت اولیه $S_1$ را مشاهده کن}
\‌به‌ازای{برای $t=1...T$}
\دستور{عمل 
	$a_t = \mu(s_t; \theta_\mu) + \mathbb{N}_t$
	 را بر اساس خط‌مشی فعلی و نویز اکتشاف، انتخاب کن و حالت بعدی $S_{t+1}$} و پاداش $R_t$ را مشاهده کن.
 \دستور{تجربه 
 	$(s_t, a_t, r_t, s_{t+1})$
 	 را در انبار تجربه $R$ ذخیره کن}
  \دستور{یک نمونه به اندازه $N$ از تجربه‌های $(s_i, a_i, r_i, s_{i+1})$ از انبار تجربه $R$ انتخاب کن }
  \دستور{وزن‌های منتقد $\theta_Q$ را با درنظر گرفتن تابع هزینه 
  	$L = \dfrac{1}{N} \sum_i(y_i - Q(S_i, A_i; \theta_Q))^2$
  	 به‌روزرسانی کن}
   \دستور{وزن‌های بازیگر $\theta_\mi$ را با استفاده از گرادیان خط‌مشی نمونه 
   	$$\nabla_{\theta_\mu} J \approx \dfrac{1}{N} \sum_{i} \nabla_a Q(s,a;\theta_Q) |_{s=s_i, a=\mu(s_i)} \nabla_{\thata_\mu} \mu(s;\theta_\mu) |_{S_i}$$
   	 به‌روزرسانی کن}
   \دستور{وزن‌های توابع هدف را به شکل 
   	$$\theta_{Q'} = \tau \theta_Q + (1-\tau) \theta_{Q'} \\ \theta_{\mu'} = \tau \theta_\mu + (1-\tau) \theta_{\mu'}$$
   	 به‌روزرسانی کن}
%\پایان‌به‌ازای
%\پایان‌به‌ازای
\پایان{الگوریتم}

\subsection{مقایسه روش بهینه‌سازی خط‌مشی و Q-learning}
نقطه قوت اصلی روش‌های بهینه‌سازی خط‌مشی، اصولی بودن آنهاست ، به این معنا که شما مستقیماً چیزی که می‌خواهید را بهینه‌سازی می‌کنید. در نتیجه این روش‌ها قابل اتکا و باثبات هستند. در مقابل، روش‌های 
\rl{Q-learning}
با یادگیری تابع Q، مقیاس عملکرد را به طور غیر مستقیم بهینه می‌کند. حالت‌های زیادی برای این نوع یادگیری وجود دارد که به شکست منتهی می‌شود، بنابراین این روش‌ها  ثبات کمتری دارند 
\cite{suttonbook}.
روش‌های 
\rl{Q-learning}
 می‌توانند از داده‌ها به طور موثرتری نسبت به تکنیک‌های بهینه‌سازی خط‌مشی استفاده کنند.
\paragraph{تعامل بین بهینه‌سازی خط‌مشی و Q-learning}
 بهینه‌سازی خط‌مشی و  
\lr{Q-learning}
ناسازگار نیستند (به نظر می‌رسد تحت برخی شرایط، معادل آن باشد) و طیف وسیعی از الگوریتم‌ها وجود دارد که بین دو حد این طیف وجود دارند. الگوریتم‌هایی که در این طیف قرار دارند قادرند از نقاط قوت  طرفین طیف استفاده کنند.
  به طور مثال، 
 \lr{DDPG}الگوریتمی‌است که  یک خط‌مشی قطعی و یک تابع Q را یاد می‌گیرد،
 به‌طوری‌که از هریک  برای بهبود دیگری استفاد می‌کند. روش
 \lr{SAC}،
   از خط‌مشی‌های تصادفی، تنظیم آنتروپی  \LTRfootnote{entropy regularization}و چند ترفند دیگر برای  یادگیری و کسب امتیاز بالاتر از 
\lr{DDPG}
    در محک‌های استاندارد
\LTRfootnote{Standard Benchmark}
    استفاده می‌کند.

%\قسمت{روش‌های مبتنی بر مدل}

%Unlike model-free RL, there aren’t a small number of easy-to-define clusters of methods for model-based RL: there are many orthogonal ways of using models. We’ll give a few examples, but the list is far from exhaustive. In each case, the model may either be given or learned.
%
%Background: Pure Planning. The most basic approach never explicitly represents the policy, and instead, uses pure planning techniques like model-predictive control (MPC) to select actions. In MPC, each time the agent observes the environment, it computes a plan which is optimal with respect to the model, where the plan describes all actions to take over some fixed window of time after the present. (Future rewards beyond the horizon may be considered by the planning algorithm through the use of a learned value function.) The agent then executes the first action of the plan, and immediately discards the rest of it. It computes a new plan each time it prepares to interact with the environment, to avoid using an action from a plan with a shorter-than-desired planning horizon.
%
%The MBMF work explores MPC with learned environment models on some standard benchmark tasks for deep RL.
%Expert Iteration. A straightforward follow-on to pure planning involves using and learning an explicit representation of the policy, \pi_{\theta}(a|s). The agent uses a planning algorithm (like Monte Carlo Tree Search) in the model, generating candidate actions for the plan by sampling from its current policy. The planning algorithm produces an action which is better than what the policy alone would have produced, hence it is an “expert” relative to the policy. The policy is afterwards updated to produce an action more like the planning algorithm’s output.
%
%The ExIt algorithm uses this approach to train deep neural networks to play Hex.
%AlphaZero is another example of this approach.
%Data Augmentation for Model-Free Methods. Use a model-free RL algorithm to train a policy or Q-function, but either 1) augment real experiences with fictitious ones in updating the agent, or 2) use only fictitous experience for updating the agent.
%
%See MBVE for an example of augmenting real experiences with fictitious ones.
%See World Models for an example of using purely fictitious experience to train the agent, which they call “training in the dream.”
%Embedding Planning Loops into Policies. Another approach embeds the planning procedure directly into a policy as a subroutine—so that complete plans become side information for the policy—while training the output of the policy with any standard model-free algorithm. The key concept is that in this framework, the policy can learn to choose how and when to use the plans. This makes model bias less of a problem, because if the model is bad for planning in some states, the policy can simply learn to ignore it.
%
%See I2A for an example of agents being endowed with this style of imagination.

%\subsection{روش مدل جهان}
%\قسمت{روش مدل جهان}
%Our world model can be trained quickly in an unsupervised manner to learn a compressed spatial and temporal representation of the environment. By using features extracted from the world model as inputs to an agent, we can train a very compact and simple policy that can solve the required task. We can even train our agent entirely inside of its own dream environment generated by its world model, and transfer this policy back into the actual environment.
%مدل جهانی
%\LTRfootnote{World Models}
% می‌تواند به سرعت و به روشی بدون نظارت آموزش ببیند تا یک بازنمایی از محیط را بیاموزد. سپس با استفاده از ویژگی‌های استخراج شده از مدل جهان به عنوان ورودی به یک عامل، می‌توان یک خط‌مشی ساده و فشرده را آموخت که می‌تواند وظیفه مورد نیاز را حل کند. حتی می‌توانیم عامل را کاملاً در داخل محیط رویایی خود که توسط مدل جهانی آن ایجاد شده، آموزش دهیم و این خط‌مشی آموخته شده را به محیط واقعی انتقال دهیم. در بسیاری از مسائل یادگیری تقویتی مبتنی بر مدل، عامل به مدل قدرتمندی از دینامیک محیط دسترسی دارد.
% \\اکثر رویکردهای مبتنی بر مدل موجود در یادگیری تقویتی، مدلی از محیط را یاد می‌گیرند، اما همچنان در محیط واقعی آموزش می‌بینند. در این روش، ما همچنین می‌توانیم یک محیط  مصنوعی را کاملاً جایگزین محیط واقعی کنیم و خط‌مشی عامل خود را فقط در داخل محیط مصنوعی آموزش دهیم و درنهایت خط‌مشی آموخته شده را به محیط واقعی انتقال دهیم.

%We present a simple model inspired by our own cognitive system. In this model, our agent has a visual sensory component that compresses what it sees into a small representative code. It also has a memory component that makes predictions about future codes based on historical information. Finally, our agent has a decision-making component that decides what actions to take based only on the representations created by its vision and memory components.