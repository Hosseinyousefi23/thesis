
% -------------------------------------------------------
%  Abstract
% -------------------------------------------------------


\pagestyle{empty}

\شروع{وسط‌چین}
\مهم{چکیده}
\پایان{وسط‌چین}
\بدون‌تورفتگی

 در این پایان نامه، به معرفی برخی از روش‌های مبتنی بر یادگیری تقویتی در حالت گسسته می‌پردازیم و الگوریتم‌های مهم آن را مرور خواهیم‌کرد. 
هدف ما این بود که به روش‌های موجود ساختاری یکپارچه و رسمی ببخشیم،
 بدون اینکه از دقت ریاضی یا قابل فهم بودن آن‌ها کاسته شود.
 مشکلات مربوط به عدم یکپارچگی در نشانه‌گذاری را اصلاح کردیم.
 تعاریف و مفاهیمی که در همه روش‌ها حضور دارند را جداگانه توضیح دادیم و مفاهیمی که تنها در یک روش، مورد استفاده قرار می‌گیرند را در قسمت مربوط به خودش توضیح دادیم.
همچنین در مواقع لزوم، برای روشن شدن مطلب، از مثال استفاده کردیم.
در میان روش\nf های کلاسیک، به روش‌های برنامه‌ریزی پویا، شامل تکرار خط‌مشی و تکرار ارزش، روش‌های تفاوت زمانی، روش‌های مبتنی بر خط‌مشی مانند گرادیان خط‌مشی کلاسیک پرداختیم.  در رویکرد مبتنی بر خط\nf مشی، به الگوریتم\nf های
\lr{Actor-Critic},
\lr{TRPO}
و
\lr{PPO}
و در رویکرد مبتنی بر ارزش، روش\nf های
\lr{Q-learning}
و
\lr{C51}
را معرفی کردیم. همچنین روش‌های میانی 
\lr{DDPG}
و
\lr{SAC}
که از ایده‌های هر دو رویکرد، استفاده می‌کنند را بررسی کردیم.
در انتها برای جمع‌بندی، برخی از مقایسه‌هایی که از این روش‌ها صورت گرفته را آورده‌ایم.


\پرش‌بلند
\بدون‌تورفتگی \مهم{کلیدواژه‌ها}: 
یادگیری تقویتی گسسته، برنامه\nf ریزی پویا، یادگیری ماشین، یادگیری ژرف
\صفحه‌جدید
